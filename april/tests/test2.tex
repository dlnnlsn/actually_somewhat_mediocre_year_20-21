\documentclass{article}

\usepackage{mathtools,amsfonts}
\usepackage{enumitem}
\usepackage{fullpage}
\usepackage{fancyvrb}
\usepackage{hyperref}


\begin{document}
\thispagestyle{empty}

\begin{center}
  \textbf{\Large Test 2}
  \\ \vspace{1em}
  \textbf{\large April Camp 2021}
  \\ \vspace{1em}
  \textbf{\large Time: $4\frac{1}{2}$ hours}
\end{center}

\vspace{24pt}

\begin{enumerate}[itemsep=18pt]

\item


\item % IMOSL 2020 N2
For each prime number $p$, there is a kingdom of $p$-Landia consisting of $p$ islands numbered $1, 2, \dotsc, p$.
Two distinct islands numbered $m$ and $n$ are connected by a bridge if and only if $p$ is a divisor of $(m^2-n+1) (n^2-m+1)$.
The bridges may pass over each other, but cannot intersect.
Prove that for infinitely many $p$ there are two islands in $p$-Landia which are not connected by a chain of bridges.


\item % IMOSL 2020 A5
A magician intends to perform the following trick, where $n$ is a positive integer.
She announces $2n$ real numbers $x_1 < x_2 < \dotsb < x_{2n}$ to the audience.
A member of the audience then secretly chooses a polynomial $P(x)$ of degree $n$ with real coefficients, computes the $2n$ values $P(x_1), \dotsc, P(x_{2n})$, and writes down these $2n$ values on the blackboard in non-decreasing order.
After that the magician announces the secret polynomial to the audience.

Can the magician find a strategy to perform such a trick?

\end{enumerate}


\vfill
\begin{itemize}
	\item Submit your solutions at \url{https://forms.gle/uhMSLew7qTQ9Qbqr6}.
	\item Submit each question in a single separate PDF file (with multiple pages if necessary).
	\item If you take photographs of your work, use a document scanner such as Office Lens to convert to PDF.
	\item If you have multiple PDF files for a question, combine them using software such as PDFsam.
\end{itemize}

\vfill
% ASCII art
\centering
\begin{BVerbatim}
                         __
           ---_ ...... _/_ -
          /  .      ./ .'*\ \
          : '         /__-'   \.
         /                      )
       _/                  >   .'
     /   .   .       _.-" /  .'
     \           __/"     /.'/|
       \ '--  .-" /     //' |\|
        \|  \ | /     //_ _ |/|
         `.  \:     //|_ _ _|\|
         | \/.    //  | _ _ |/| ASH
          \_ | \/ /    \ _ _ \\\
              \__/      \ _ _ \|\
\end{BVerbatim}

\end{document}
