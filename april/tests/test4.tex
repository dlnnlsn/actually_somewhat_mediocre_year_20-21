\documentclass{article}

\usepackage{mathtools,amsfonts}
\usepackage{enumitem}
\usepackage{fullpage}
\usepackage{fancyvrb}
\usepackage{hyperref}


\begin{document}
\thispagestyle{empty}

\begin{center}
  \textbf{\Large Test 4}
  \\ \vspace{1em}
  \textbf{\large April Camp 2021}
  \\ \vspace{1em}
  \textbf{\large Time: $4\frac{1}{2}$ hours}
\end{center}

\vfill

\begin{enumerate}[leftmargin=0pt, itemsep=12pt]

\item % China Girls Maths Olympiad 2011, P7
We have $n$ boxes labelled $B_1, B_2, \dots, B_n$. Initially there are $n$ balls in total in these $n$ boxes. (There may be some boxes with more than one ball, and some boxes that are empty.) One each turn, we are allowed to perform one of three operations:
\begin{itemize}
  \item If there is at least $1$ ball in box $B_1$, we can remove this ball from box $B_1$ and add a ball to box $B_2$.
  \item If there are at least $2$ balls in box $B_i$ where $1 < i < n$, then we can remove $2$ balls from box $B_i$, and add $1$ ball to each of boxes $B_{i - 1}$ and $B_{i + 1}$.
  \item If there is at least $1$ ball in box $B_n$, we can remove this ball from box $B_n$ and add a ball to box $B_{n - 1}$.
\end{itemize}
Show that no matter how the balls are initially distributed, it is possible to use these operations to obtain the situation where there is exactly one ball in each box.


\item % IMOSL 2020 G1
Let $ABC$ be an isosceles triangle with $BC = CA$, and let $D$ be a point on segment $AB$ such that $AD < DB$.
Let $P$ and $Q$ be points on segments $BC$ and $CA$ respectively such that $\angle DPB = \angle DQA = 90^\circ$.
Let the perpendicular bisector of $PQ$ meet line segment $CQ$ at $E$, and let the circumcircles of $ABC$ and $CPQ$ meet again at point $F$, different to $C$.

Suppose that $P$, $E$, and $F$ are collinear.
Prove that $\angle ACB = 90^\circ$.


\item % IMOSL 2020 N4
For any odd prime $p$ and any integer $n$, let $d_p(n) \in \{0, 1, 2, \dotsc, p-1\}$ denote the remainder when $n$ is divided by $p$.
We say that $(a_0, a_1, a_2, \dotsc)$ is a $p$-sequence if $a_0$ is a positive integer coprime to $p$ and $a_{n+1} = a_n +d_p(a_n)$ for $n \geq 0$.
\begin{enumerate}[label=(\alph*)]
  \item Do there exist infinitely many primes $p$ for which there exist $p$-sequences $(a_0, a_1, \dotsc)$ and $(b_0, b_1, \dotsc)$ such that $a_n > b_n$ for infinitely many $n$ and $a_n < b_n$ for infinitely many $n$?
  \item Do there exist infinitely many primes $p$ for which there exist $p$-sequences $(a_0, a_1, \dotsc)$ and $(b_0, b_1, \dotsc)$ such that $a_0 < b_0$ but $a_n > b_n$ for all $n \geq 1$?
\end{enumerate}

\end{enumerate}


\vfill
\vfill
\begin{itemize}
	\item Submit your solutions at \url{https://forms.gle/uhMSLew7qTQ9Qbqr6}.
	\item Submit each question in a single separate PDF file (with multiple pages if necessary).
	\item If you take photographs of your work, use a document scanner such as Office Lens to convert to PDF.
	\item If you have multiple PDF files for a question, combine them using software such as PDFsam.
\end{itemize}

\vfill
% ASCII art
\centering
\tiny
\begin{BVerbatim}
            /^\/^\
          _|__|  O|
 \/     /~     \_/ \
  \____|__________/  \
         \_______      \
                 `\     \                 \
                   |     |                  \
                  /      /                    \
                 /     /                       \\
               /      /                         \ \
              /     /                            \  \
            /     /             _----_            \   \
           /     /           _-~      ~-_         |   |
          (      (        _-~    _--_    ~-_     _/   |
           \      ~-____-~    _-~    ~-_    ~-_-~    /
             ~-_           _-~          ~-_       _-~   - jurcy -
                ~--______-~                ~-___-~
\end{BVerbatim}

\end{document}
