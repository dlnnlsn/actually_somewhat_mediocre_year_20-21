\documentclass{article}

\usepackage{mathtools,amsfonts}
\usepackage{enumerate}
\usepackage{fullpage}
\usepackage{fancyvrb}
\usepackage{hyperref}


\begin{document}
\thispagestyle{empty}

\begin{center}
  \textbf{\Large Advanced February Monthly Assignment Solutions}
\end{center}

\vspace{12pt}

\begin{enumerate}[1.]

\vspace{24pt}
\item % DB-2014-1
{\itshape Prove that among the first $30000$ positive integers there are at least $22000$ composite numbers.}

Among the first $30000$ positive integers, there are $\frac{30000}{2} = 15000$ multiples of $2$, $10000$ multiples of $3$, and $6000$ multiples of $5$. There are then at most $15000 + 10000 + 6000 = 31000$ multiples of $2$, $3$, or $5$. But this counts each multiples of $6$, $10$, and $15$ twice, so we subtract the $5000 + 3000 + 2000 = 10000$ multiples of $6$, $10$, or $15$ to arrive at at least $31000 - 10000 = 21000$ multiples of $2$, $3$, or $5$. But now we have added each multiple of $30$ twice, but subtracted them $3$ times, so we add $\frac{30000}{30} = 1000$ to account for the multiples of $30$.

We thus see that among the first $30000$ positive integers, there are $22000$ that are a multiple of $2$, $3$, or $5$. These are all composite except for $2$, $3$, or $5$ themselves, so we need only find $3$ more composite numbers below $30000$ that are not divisible by any of these. The numbers $49$, $77$, and $121$ will do.


\vspace{24pt}
\item % Ireland 2018 Q9
{\itshape Suppose $a,b,c > 0$ and $\sqrt{a-b} +\sqrt{a-c} > \sqrt{b+c}$. Prove that $a > \dfrac{3}{4} (b+c)$.}

Suppose that $a \leq \dfrac{3}{4} (b + c)$. We will show that $\sqrt{a-b} +\sqrt{a-c} \leq \sqrt{b+c}$. Since $a \leq \dfrac{3}{4} (b + c)$, we have that
\[
	\sqrt{a-b} +\sqrt{a-c} \leq \sqrt{\frac{3}{4}(b + c) - b} + \sqrt{\frac{3}{4}(b + c) - c} = \frac{1}{2} \left(\sqrt{3c - b} + \sqrt{3b - c}\right).
\]

Now
\begin{align*}
	& \dfrac{1}{2} \left(\sqrt{3c - b} + \sqrt{3b - c}\right)  \leq \sqrt{b + c} \\
	\iff & (3c - b) + (3b - c) + 2\sqrt{(3c - b)(3b - c)}  \leq 4(b + c) \\
	\iff & (3c - b)(3b - c) \leq (b + c)^2 \\
	\iff & 10bc - 3b^2 - 3c^2 \leq b^2 + 2bc + c^2 \\
	\iff & 4(b - c)^2 \geq 0
\end{align*}
and so we are done.


\vspace{24pt}
\item % NH-2006-1
{\itshape Let $ABCD$ be a square.
Points $P$ and $Q$ lie on the segments $AD$ and $DC$ such that $\angle PBQ = 45^\circ$.
Prove that $BP$ bisects $\angle APQ$ and $BQ$ bisects $\angle AQP$.}


\vspace{24pt}
\item % French Number Theory Training Assignment
{\itshape Let $n$ be a positive integer. Show that there is a positive integer $m$ such that $\varphi(m) = n!$, where $\varphi$ denotes the Euler phi function.}


\vspace{24pt}
\item % China North MO 2020 Advanced Level P1
{\itshape Define the function $f(x) = x^2 + \sin(x)$ (where $x$ is in radians in this context). Furthermore, let $\{a_n\}$ be a sequence with $a_n \in \mathbb{R}^+$ for all $n \in \mathbb{N}$. Let $a_1 = 1$, and $f(a_n) = a_{n - 1}$ for $n \ge 2$. Prove that there exists $n \in \mathbb{N}$ such that 
$$\sum_{k = 1}^n a_i > 2021.$$}


\vspace{24pt}
\item % Ukraine 2016-2017 p54 Q10
{\itshape Consider a triangle $ABC$ with points $M$ and $N$ on $BC$ and $AB$ respectively such that $AM \perp BC$ and $CN \perp AB$, and let $AC$ and $MN$ intersect at $Y$.
Let $X$ be a point inside triangle $ABC$ such that $MBNX$ is a parallelogram.
Prove that the angle bisectors of $\angle MXN$ and $\angle MYC$ are perpendicular.}


\end{enumerate}

\end{document}
