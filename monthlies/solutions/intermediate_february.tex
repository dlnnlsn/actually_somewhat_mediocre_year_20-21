\documentclass{article}

\usepackage{mathtools,amsfonts}
\usepackage{enumerate}
\usepackage{fullpage}
\usepackage{fancyvrb}
\usepackage{hyperref}


\begin{document}
\thispagestyle{empty}

\begin{center}
  \textbf{\Large Intermediate February Monthly Assignment Solutions}
\end{center}

\vspace{12pt}

\begin{enumerate}[1.]

\vspace{6pt}
\item % Jon
{\itshape The natural number $n$ can be replaced by $ab$ if $a + b = n$, where $a$ and $b$ are natural numbers. Can the number $2021$ be obtained from $22$ after a sequence of such
replacements?}


\vspace{6pt}
\item % DB-2014-1
{\itshape Prove that among the first $30000$ positive integers there are at least $22000$ composite numbers.}

Among the first $30000$ positive integers, there are $\frac{30000}{2} = 15000$ multiples of $2$, $10000$ multiples of $3$, and $6000$ multiples of $5$. There are then at most $15000 + 10000 + 6000 = 31000$ multiples of $2$, $3$, or $5$. But this counts each multiples of $6$, $10$, and $15$ twice, so we subtract the $5000 + 3000 + 2000 = 10000$ multiples of $6$, $10$, or $15$ to arrive at at least $31000 - 10000 = 21000$ multiples of $2$, $3$, or $5$. But now we have added each multiple of $30$ twice, but subtracted them $3$ times, so we add $\frac{30000}{30} = 1000$ to account for the multiples of $30$.

We thus see that among the first $30000$ positive integers, there are $22000$ that are a multiple of $2$, $3$, or $5$. These are all composite except for $2$, $3$, or $5$ themselves, so we need only find $3$ more composite numbers below $30000$ that are not divisible by any of these. The numbers $49$, $77$, and $121$ will do.


\vspace{6pt}
\item % Salvador 2017 9th grade Q2
{\itshape Let $a$ and $b$ be positive real numbers such that $2a^2 +2b^2 = 5ab$.
If $|x|$ denotes the absolute value of $x$, calculate
\[ \left|\frac{a+b}{a-b}\right|. \]}


\vspace{6pt}
\item % Taariq, stolen from AOPS
{\itshape Triangle $ABC$ is a right angled triangle with $\angle C = 90^{\circ}$. $P$ is placed randomly inside $\triangle ABC$. What is the probability that the area of $\triangle PBC$ is less than half of the area of $\triangle ABC$?
\begin{figure}[h]
\centering
\includegraphics[width=0.5\textwidth]{../fig.jpg}
\end{figure}}


\vspace{6pt}
\item % Danielle/Taariq
{\itshape Let $c$ and $d$ be positive divisors of a natural number $n$ such that $c > d$. Prove that $$c > d + \frac{d^2}{n}.$$}


\vspace{6pt}
\item % Ireland 2018 Q9
{\itshape Suppose $a,b,c > 0$ and $\sqrt{a-b} +\sqrt{a-c} > \sqrt{b+c}$. Prove that $a > \dfrac{3}{4} (b+c)$.}


\end{enumerate}

\end{document}
