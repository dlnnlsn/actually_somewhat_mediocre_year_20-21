\documentclass{article}

\usepackage{mathtools,amsfonts,amssymb}
\usepackage{enumerate}
\usepackage{fullpage}
\usepackage{fancyvrb}


\begin{document}
\thispagestyle{empty}

\begin{center}
  \textbf{\Large Advanced Test 1 Solutions}
  % LEVEL is Senior, Intermediate or Beginner
  % NUMBER is the test number: 1, 2, etc.
  \\ \vspace{1em}
  \textbf{\large January Camp 2021}
  \\ \vspace{1em}
  \textbf{\large Time: $2\frac{1}{2}$ hours}
\end{center}

\vspace{24pt}

\begin{enumerate}[1.]

\item % Ukraine 2018-2019, 3rd round, second tour, 9th Grade, Q1
{\itshape Find all possible real numbers $k$ such that the values of $x$ satisfying
$$k(2 - k)x^2 - (k + 4)x + 6 = 0$$
are positive integers.}

Firstly, notice that if $k(2 - k) = 0$, then $x = \frac{6}{k + 4}$. The only positive integer solution of this is when $k = 2$: $x = 1$.
Assuming $k(2 - k) \ne 0$, we can use the quadratic formula. We see that the solutions of the equation in $x$ are
\begin{align*}
    x &= \frac{(k + 4) \pm \sqrt{(k + 4)^2 - 24k(2 - k)}}{2k(2 - k)} \\
    &= \frac{(k + 4) \pm \sqrt{25k^2 - 40k + 16}}{2k(2 - k)} \\
    &= \frac{(k + 4) \pm \sqrt{(5k - 4)^2}}{2k(2 - k)} \\
    &= \frac{(k + 4) \pm (5k - 4)}{2k(2 - k)}
\end{align*}
The two solutions are thus $x_1 = \frac{6k}{2k(2 - k)} = \frac{3}{2 - k}$ and $x_2 = \frac{8 - 4k}{2k(2 - k)} = \frac{2}{k}$. Now we must have that $\frac{2}{k}$ and $\frac{3}{2 - k}$ are integers simultaneously. $\frac{2}{k} \in \mathbb{Z} \iff k = \frac{2}{n}$ where $n \in \mathbb{Z}$. Hence we must have the following is an integer 
$$\frac{3}{2 - k} = \frac{3}{2 - \frac{2}{n}} = \frac{3n}{2n - 2}= \frac{3n}{2(n - 1)}$$

Having $n - 1 | 3$ yields $n \in \{-2, 0, 2, 4\}$. If $n - 1 \nmid 3$, we must have $2(n - 1) | n$. If $n > 2$, then $2(n - 1) > n$. If $n < 0$, then $2(n - 1) < n < 0$. Thus we only need to check $n \in \{0, 1, 2\}$. The values of $k$ that we get are $k \in \{-1, 1, \frac{1}{2}, 2\}$. It can then be checked that the only $k$ values that provide positive integers solutions are 
$$k \in \{\frac{1}{2}, 1, 2\}$$

\item % Ireland, 2017, Powerful Sequences, Q2
{\itshape Let $O$ be the circumcentre of $\triangle ABC$. Let $X$, $Y$ and $Z$ be the reflections of $O$ over $BC$, $CA$ and $AB$ respectively. Prove that $\triangle XYZ$ is congruent to $\triangle ABC$ and the corresponding sides are parallel.}


\item % French Olympiad Training Assignment from 2015-2016: http://maths-olympiques.fr/wp-content/uploads/2017/10/ofm-2015-2016-envoi-1.pdf
{\itshape Find the smallest non-negative integer which can not be written in the from
\[
  \frac{2^a - 2^b}{2^c - 2^d}
\]
for some positive integers $a$, $b$, $c$, and $d$.}

We have the following representations for the numbers from $0$ to $10$:
\begin{align*}
  0 & = \frac{2^1 - 2^1}{2^2 - 2^1} & 1 & = \frac{2^2 - 2^1}{2^2 - 2^1} & 2 & = 2 \times 1 = \frac{2^3 - 2^2}{2^2 - 2^1} \\
  3 & = \frac{2^3 - 2^1}{2^2 - 2^1} & 4 & = 2 \times 2 = \frac{2^4 - 2^3}{2^2 - 2^1} & 5 & = \frac{2^5 - 2^1}{2^3 - 2^1} \\
  6 & = 2 \times 3 = \frac{2^4 - 2^2}{2^2 - 2^1} & 7 & = \frac{2^4 - 2^1}{2^2 - 2^1} & 8 & = 2 \times 4 = \frac{2^5 - 2^4}{2^2 - 2^1} \\
  9 & = \frac{2^7 - 2^1}{2^4 - 2^1} & 10 & = 2 \times 5 = \frac{2^6 - 2^2}{2^3 - 2^1} & &
\end{align*}

We claim that there is no representation for the number $11$. Suppose that
\[
  2^a - 2^b = 11(2^c - 2^d).
\]
We may suppose without loss of generality that $a > b$ and $c > d$. Then the largest power of $2$ dividing the left hand side of the equation is $2^b$, and the largest power of $2$ dividing the right hand side is $2^d$. Thus we must have that $b = d$. Letting $x = a - b$ and $y = c - d$, we obtain
\[
  2^x - 1 = 11(2^y - 1).
\]
Consider this equation modulo $5$. We obtain that
\[
  2^x \equiv 2^y \pmod 5 \implies 2^{x - y} \equiv 1 \pmod 5.
\]
By trial and error or otherwise, we know that the smallest power of $2$ that leaves a remainder of $1$ when divided by $5$ is $2^4$, and so we have that $x - y \geq 4$. But then
\[
  2^x - 1 \geq 2^{y + 4} - 1 > 16(2^y - 1) > 11(2^y - 1)
\]
and so it is not possible for $2^x - 1$ to be equal to $11(2^y - 1)$.


\item % Indonesia 2012, Day 2 Q2
{\itshape Find all functions $f: \mathbb{R}^+ \rightarrow \mathbb{R}^+$ satisfying 
$$f(x + y) = f(x) + f(y) + \frac{1}{2021} $$
Where $\mathbb{R}^+$ is the set of positive real numbers. }

Assume that such a function $f$ exists. Consider the function $g: \mathbb{R}^+ \rightarrow \mathbb{R}$ defined by 
$$g(x) = f(x) + \frac{1}{2021}$$
Rewriting the function equation, in terms of $g$, we get. 
$$g(x + y) = g(x) + g(y)$$
Clearly, $g(nx) = ng(x)$ for $n \in \mathbb{Z}$. Letting $x = \frac{1}{n}$, we get $g(1) = ng(\frac{1}{n}) \iff g(\frac{1}{n}) = \frac{g(1)}{n}$. Now no matter what value $g(1)$ takes, we can find a large enough $n$ such that $g(\frac{1}{n}) = \frac{g(1)}{n} < \frac{1}{2021}$. Thus, we must have $f(\frac{1}{n}) + \frac{1}{2021} < \frac{1}{2021} \implies f(\frac{1}{n}) < 0$. Since the codomain is the positive real numbers, we cannot have $f$ returning a negative value. This is thus a contradiction.

Therefore, there are no functions $f$ satisfying the functional equation.


\item % Jon
{\itshape Jon has a collection of weights with different positive integer values. Is it possible that there are exactly $2020$ ways to choose some of these distinct weights such that their total weight is $2020$?}

\end{enumerate}


\vfill
% ASCII art
\centering
\begin{BVerbatim}
      ,~~.
     (  6 )-_,
(\___ )=='-'
 \ .   ) )
  \ `-' /    
~'`~'`~'`~'`~
\end{BVerbatim}

\end{document}
