\documentclass{article}

\usepackage{mathtools,amsfonts,amssymb}
\usepackage{enumerate}
\usepackage{fullpage}
\usepackage{fancyvrb}


\begin{document}
\thispagestyle{empty}

\begin{center}
  \textbf{\Large Advanced Test 2 Solutions}
  % LEVEL is Senior, Intermediate or Beginner
  % NUMBER is the test number: 1, 2, etc.
  \\ \vspace{1em}
  \textbf{\large January Camp 2021}
\end{center}

\vspace{24pt}

\begin{enumerate}[1.]

  \item % Danielle from Ukrainian MO
  {\itshape A positive integer $N$ has exactly $2021$ positive divisors (including $1$ and $N$ itself), and it is divisible by $2021$.
  Prove that $N$ is not divisible by $2021^{43}$.}

  We recall that if
  \[
    N = {p_1}^{a_1} {p_2}^{a_2} \cdots {p_k}^{a_k}
  \]
  is the prime factorisation of $N$, then the number of divisors of $N$ is given by
  \[
    (a_1 + 1)(a_2 + 1) \cdots (a_k + 1).
  \]

  We thus investigate the solutions to
  \[
    (a_1 + 1)(a_2 + 1) \cdots (a_k + 1) = 2021.
  \]
  We know that $2021 = 43 \times 47$, which are both prime, and so the only ways of factorising $2021$ as a product of some number of integers are $2021$ and $43 \times 47$.

  Since $N$ is divisible by $2021$, $N$ has at least the two primes factors $43$ and $47$, and so it is the second factorisation that is relevant: we must have that
  \begin{align*}
    a_1 + 1 & = 43 & \text{ and } && a_2 + 1 & = 47
  \end{align*}
  or vice versa. Thus the only options for $N$ are $43^{42} \times 47^{46}$, or $43^{46} \times 47^{42}$, neither of which is divisible by $2021^{43} = 43^{43} \times 47^{43}$.
  
  
  \item % Ralph: found in one of the books. Can't remember which one.
  {\itshape Let $a$, $b$, $c$, $x$, $y$ and $z$ be positive real numbers with $a + b + c = x + y + z$.
  Prove that 
  \[ \frac{a}{x + y} + \frac{b}{y + z} + \frac{c}{z + x} + \frac{x}{a + b} + \frac{y}{b + c} + \frac{z}{c + a} > 2. \]}

  Increasing the value of each of the denominators decreases the value of each fraction, and so
  \begin{align*}
    \frac{a}{x + y} & + \frac{b}{y + z} + \frac{c}{z + x} + \frac{x}{a + b} + \frac{y}{b + c} + \frac{z}{c + a} \\
    & > \frac{a}{x + y + z} + \frac{b}{y + z + x} + \frac{c}{z + x + y} + \frac{x}{a + b + c} + \frac{y}{b + c + a} + \frac{z}{c + a + b} \\
    & = \frac{a + b + c}{x + y + z} + \frac{x + y + z}{a + b + c} \\
    & = 2.
  \end{align*}
  
  
  \item % Ireland Book 2017 P15 (orange book)
  {\itshape Let circles $\Gamma_1$ and $\Gamma_2$ intersect at $A$ and $B$. One of the tangents to $\Gamma_1$ and $\Gamma_2$ touches them at $P$ and $Q$ respectively. Let line $AB$ meet the circumcircle of $PQA$ at $C$. Join $CP$ and $CQ$ and extend both to meet $\Gamma_1$ and $\Gamma_2$ at $F$ and $E$ respectively. Prove that the quadrilateral $PQFE$ is cyclic.}
  
  
  \item % APMO 2007
  {\itshape Let $K$ be a set of nine different positive integers which only have $2027$ and $2029$ as prime factors.
  Show that there are three distinct integers $a$, $b$, and $c$ in $K$ such that $\sqrt[3]{abc}$ is an integer.}
  
  
  \item % Romania TST 2011, P12
  {\itshape Prove that there are infinitely many $n \in \mathbb{N}$ such that there exists a $d \in \mathbb{N}$ with both $d$ and $d + n$ being factors of $n^2 + 1$.}
  
  \begin{description}
    \item[Solution 1] Note that $d = n = 1$ provides a solution. Suppose that $k$ and $m$ are such that $k \mid m^2 + 1$ and $k + m \mid m^2 + 1$. Let
    \begin{align*}
      d & = k + m & \text{ and } && n & = m + \frac{m^2 + 1}{k}.
    \end{align*}
    We claim that $d \mid n^2 + 1$ and $d + n \mid n^2 + 1$. We first show that $d \mid n^2 + 1$. Since $\gcd(d, k) = \gcd(m, k) = 1$, it is enough to show that
    \[
      k^2 (n^2 + 1) = (m^2 + km + 1)^2 + k^2
    \]
    is divisible by $d = k + m$. We note that
    \[
      (m^2 + km + 1)^2 + k^2 \equiv k^2 + 1 \equiv (-m)^2 + 1 \equiv 0 \pmod{k + m}
    \]
    since $k + m \mid m^2 + 1$ by assumption. We now show that $n + d \mid n^2 + 1$. We know that $n^2 \equiv (-d)^2 + 1 \pmod{n + d}$, so we can instead show that $n + d \mid d^2 + 1$. We note that
    \[
      k(n + d) = (km + m^2 + 1) + k(k + m) = k^2 + 2km + m^2 + 1 = (k + m)^2 + 1 = d^2 + 1.
    \]
    and so $d^2 + 1$ is divisible by $n + d$, as claimed.

    We see that if $k$ and $m$ are such that $k \mid m^2 + 1$ and $k + m \mid m^2 + 1$, then the values of $d$ and $n$ given above also provide a solution to the problem. Since $n > m$, we obtain infinitely many solutions in this way.

    \item[Solution 2] We recall \emph{Cassini's Identity}
    \[
      F_{n + 1} F_{n - 1} - F_n^2 = (-1)^n
    \] 
    where $F_k$ is the $k^\text{th}$ Fibonacci number. Considering the even terms gives us
    \[
      F_{2n}^2 + 1 = F_{2n + 1} F_{2n - 1} = F_{2n - 1} (F_{2n} + F_{2n - 1})
    \]
    and so $n = F_{2n}$ and $d = F_{2n - 1}$ provides infinitely many solutions to the problem.
  \end{description}

  \end{enumerate}
  
  \vfill
  % ASCII art
  \centering
  \begin{BVerbatim}
    .-~~~~-.
   {  o     }
   /       /
  `--r'   {    ,___.-',
    /      `-~         ',
   {                    '
    \                  /
     \                /
    ~ ~~~~~~~~~~~~~~~~~ ~
   ~ ~ ~ ~ ~ ~ ~ ~ ~ ~ ~ ~
    ~ ~ ~ ~ ~ ~ ~ ~ ~ ~ ~
       ~ ~ ~ ~ ~ ~ ~ ~
  \end{BVerbatim}
  
  \end{document}
  