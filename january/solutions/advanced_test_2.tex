\documentclass{article}

\usepackage{mathtools,amsfonts,amssymb}
\usepackage{enumerate}
\usepackage{fullpage}
\usepackage{fancyvrb}


\begin{document}
\thispagestyle{empty}

\begin{center}
  \textbf{\Large Advanced Test 2 Solutions}
  % LEVEL is Senior, Intermediate or Beginner
  % NUMBER is the test number: 1, 2, etc.
  \\ \vspace{1em}
  \textbf{\large January Camp 2021}
\end{center}

\vspace{24pt}

\begin{enumerate}[1.]

  \item % Danielle from Ukrainian MO
  {\itshape A positive integer $N$ has exactly $2021$ positive divisors (including $1$ and $N$ itself), and it is divisible by $2021$.
  Prove that $N$ is not divisible by $2021^{43}$.}
  
  
  \item % Ralph: found in one of the books. Can't remember which one.
  {\itshape Let $a$, $b$, $c$, $x$, $y$ and $z$ be positive real numbers with $a + b + c = x + y + z$.
  Prove that 
  \[ \frac{a}{x + y} + \frac{b}{y + z} + \frac{c}{z + x} + \frac{x}{a + b} + \frac{y}{b + c} + \frac{z}{c + a} > 2. \]}
  
  
  \item % Ireland Book 2017 P15 (orange book)
  {\itshape Let circles $\Gamma_1$ and $\Gamma_2$ intersect at $A$ and $B$. One of the tangents to $\Gamma_1$ and $\Gamma_2$ touches them at $P$ and $Q$ respectively. Let line $AB$ meet the circumcircle of $PQA$ at $C$. Join $CP$ and $CQ$ and extend both to meet $\Gamma_1$ and $\Gamma_2$ at $F$ and $E$ respectively. Prove that the quadrilateral $PQFE$ is cyclic.}
  
  
  \item % APMO 2007
  {\itshape Let $K$ be a set of nine different positive integers which only have $2027$ and $2029$ as prime factors.
  Show that there are three distinct integers $a$, $b$, and $c$ in $K$ such that $\sqrt[3]{abc}$ is an integer.}
  
  
  \item % Romania TST 2011, P12
  {\itshape Prove that there are infinitely many $n \in \mathbb{N}$ such that there exists a $d \in \mathbb{N}$ with both $d$ and $d + n$ being factors of $n^2 + 1$.}
  
  \end{enumerate}
  
  \vfill
  % ASCII art
  \centering
  \begin{BVerbatim}
    .-~~~~-.
   {  o     }
   /       /
  `--r'   {    ,___.-',
    /      `-~         ',
   {                    '
    \                  /
     \                /
    ~ ~~~~~~~~~~~~~~~~~ ~
   ~ ~ ~ ~ ~ ~ ~ ~ ~ ~ ~ ~
    ~ ~ ~ ~ ~ ~ ~ ~ ~ ~ ~
       ~ ~ ~ ~ ~ ~ ~ ~
  \end{BVerbatim}
  
  \end{document}
  