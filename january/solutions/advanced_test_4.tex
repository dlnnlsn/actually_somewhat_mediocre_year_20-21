\documentclass{article}

\usepackage{mathtools,amsfonts}
\usepackage{enumerate}
\usepackage{fullpage}
\usepackage{fancyvrb}
\usepackage{hyperref}
\usepackage{booktabs}


\begin{document}
\thispagestyle{empty}

\begin{center}
  \textbf{\Large Advanced Test 4 Solutions}
  % LEVEL is Senior, Intermediate or Beginner
  % NUMBER is the test number: 1, 2, etc.
  \\ \vspace{1em}
  \textbf{\large January Camp 2021}
  \\ \vspace{1em}
  \textbf{\large Time: $2\frac{1}{2}$ hours}
\end{center}

\vspace{12pt}

\begin{enumerate}[1.]

\item % Tim
{\itshape Consider a $3\times3\times3$ 3-dimensional chess cube with some hyperrooks.
Hyperrooks can move along any direction parallel to an edge of the cube (like a normal rook, but also up and down).
What is the maximum number of hyperrooks you can place in the chess cube without any of them attacking each other?}

First, we prove that 9 hyperrooks is the maximum amount of hyperrooks you can place on a chess cube. Consider the 9 columns (in the up-down direction) of the chess cube. If a column contains 2 hyperrooks, the hyperrooks would attack each other. Therefore, each column has at most 1 hyperrook. Since there are 9 columns, there are at most 9 hyperrooks. We show that placing 9 hyperrooks is possible by construction.

Construction for 9 rooks:
\begin{table}[h]
\centering
\begin{tabular}{| c | c| c |}
\multicolumn{3}{c}{Top layer}\\
\hline
R& &\\
\hline
&R&\\
\hline
&&R\\
\hline
\end{tabular}
\quad
\begin{tabular}{| c | c| c |}
\multicolumn{3}{c}{Middle layer}\\
\hline
&R &\\
\hline
&&R\\
\hline
R&&\\
\hline
\end{tabular}
\quad
\begin{tabular}{| c | c| c |}
\multicolumn{3}{c}{Bottom layer}\\
\hline
& &R\\
\hline
R&&\\
\hline
&R&\\
\hline
\end{tabular}
\end{table}

\item % Sharp Maths Competition, Easter Egg for Liam
{\itshape Find all positive integers $a$, $b$ and $c$ satisfying 
$$a + b - c = 14$$
$$a^2 + b^2 - c^2 = 14.$$}


\item % ELMO 2013 Shortlist C3
{\itshape You are given nine real numbers, $a_1$, $a_2$, $\cdots$, $a_9$ with an average of $m$. What is the minimum possible number of triples $(i, j, k)$ with $1 \le i < j < k \le 9$ and $a_i + a_j + a_k \ge 3m$?}


\item % Macedonia 2018, Q5
{\itshape Let $ABC$ be an acute-angled triangle with orthocentre $H$. Let the point $H'$ be the reflection of $H$ over $AB$. Let $N$ be the intersection of $HH'$ and $AB$. The circumcircle of $\triangle ANH'$ intersects $AC$ again at $M$. The circumcircle of $\triangle BNH'$ intersects $BC$ again at $P$. Show that the points $M$, $N$ and $P$ are collinear.}


\item % Rioplatense Mathematical Olympiad, Level 3, Day 2, Q1
{\itshape Find all functions $f: \mathbb{R} \rightarrow \mathbb{R}$ such that 
$$f(xy) = \max\{f(x + y), f(x)f(y)\} $$
for all $x$, $y \in \mathbb{R}$.}

Substituting $(x, y) = (0, 0)$, we get $f(0) = \max\{f(0), f(0)^2\}$, so we must have 
$$f(0) \ge f(0)^2 \implies f(0) \in [0, 1]$$ 
Now we substitute $(x, y) = (x, 0)$ giving $f(0) = \max\{f(x), f(x)f(0)\}$. Lastly, substituting $(x, y) = (x, -x)$, we get $f(-x^2) = \max\{f(0), f(x)f(-x)\}$. If we let $f(0) = a$, then these three identities read as follows
\begin{align}
a &\in [0, 1] \label{a-cond}\\
a &= \max\{f(x), af(x)\} \label{f-cond}\\
f(-x^2) &= \max\{a, f(x)f(-x)\} \label{square-cond}
\end{align}
Identity (\ref{a-cond}) with (\ref{f-cond}) tells us that if $f(k) \ge 0$, then $f(k) \ge af(k)$, so $f(k) = a$. We shall now consider two separate cases:
\begin{itemize}
  \item Case 1: $a = 0$ \\
  Identity (\ref{f-cond}) gives $0 = \max\{f(x), 0\} \; \forall x \in \mathbb{R} \implies f(x) \le 0 \; \forall x \in \mathbb{R}$. Identity (\ref{square-cond}) gives $f(-x^2) = 0$ and $f(x)f(-x) \le 0$. Thus, for $k \le 0$, we must have $f(k) = 0$. Consider now the substitution $(x,y) = (x, -1)$; we get 
  $$f(-x) = \max\{f(x - 1), f(x)f(-1)\} = \max\{f(x - 1), 0\} = 0 \; \forall x \in \mathbb{R}$$ 
  so in particular we then have $f(x) = 0 \; \forall x \in \mathbb{R}$.
  \item Case 2: $a \in (0, 1]$ \\
  Suppose $f(k) \le 0$. We must then have $f(k) \le af(k) \implies af(k) = a$ by Identity (\ref{f-cond}). Solving this we get 
  $$a(f(k) - 1) = 0 $$
  So either $a = 0$ or $f(k) = 1$. The assumption in this case was that $a > 0$ and $f(k) \le 0$, so we have found a contradiction. Hence there cannot be any $k \in \mathbb{R}$ such that $f(k) \le 0$. This means that $f(x) > 0 \; \forall x \in \mathbb{R}$, but we already have shown that this must give $f(x) = a \; \forall x \in \mathbb{R}$.
\end{itemize}
In summary: $f(0) = 0 \implies f(x) = 0 \; \forall x \in \mathbb{R}$ and $f(0) = a \in (0, 1] \implies f(x) = a \; \forall x \in \mathbb{R}$. Hence we can write this concisely as $f(x) = a \; \forall x \in \mathbb{R}$ where $a \in [0, 1]$.

Check:
\begin{align*}
  \text{LHS} &= f(xy) = a \\
  \text{RHS} &= \max\{f(x + y), f(x)f(y) \} = \max\{a, a^2\} = a \\
  \implies \text{LHS} &= \text{RHS}
\end{align*}

\end{enumerate}

\vfill
% ASCII art
\centering
\tiny
\begin{BVerbatim}
              .-""-.
             /      \
            /     (0 \______
            |         "_____)
            \        ,-----'
             \_    _/
              /    \
             /      \
            /        \
           /          \
          /        :   |
         /     ;   :   |
\\\     /  _.-'    :   |
 \\\\  / _'        :   |
  \\\\/ ;         :   /
   \\  ;         :   /
    \   `._`-'_.'  _/
     \     ''' _.-'
      \      / /
       \    / /
        \  /)(_______
         )(_________<
        (__________<
\end{BVerbatim}

\end{document}
