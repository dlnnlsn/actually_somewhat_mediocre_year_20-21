\documentclass{article}

\usepackage{mathtools,amsfonts,amssymb}
\usepackage{enumerate}
\usepackage{fullpage}
\usepackage{fancyvrb}
\usepackage{booktabs}

\begin{document}
\thispagestyle{empty}

\begin{center}
  \textbf{\Large Intermediate Test 2 Solutions}
  % LEVEL is Senior, Intermediate or Beginner
  % NUMBER is the test number: 1, 2, etc.
  \\ \vspace{1em}
  \textbf{\large January Camp 2021}
\end{center}

\vspace{24pt}

\begin{enumerate}[1.]

  \item % Taariq, well known, not even a problem really
  {\itshape Let $\triangle ABC$ be an isosceles triangle with $AB = AC$. Let $H$ be the orthocentre of $\triangle ABC$ and let $K$ be the reflection of $H$ across $BC$. Prove that $HBKC$ is a parallelogram.}
  
  By reflection $\angle BHC = \angle BKC$, since $AB = AC$ and the diagram is symmetric with respect to $AH$, $\angle HBK = \angle HCK$. Hence in quadrilateral $BHCK$, opposite angles are equal, making it a parallelogram.
  
  \item % Taariq, modified from AIME
  {\itshape Let $a$ and $r$ be real numbers such that
  \[ a +ar +ar^2 +ar^3 +\dotsb +ar^{2020} = 200 \qquad\text{and}\qquad a +ar +ar^2 +ar^3 +\dotsb +ar^{4041} = 380. \]
  Find the value of
  \[ a +ar +ar^2 +ar^3 +\dotsb +ar^{6062}. \]}
  
  \textbf{Solution 1}\newline

We can factorise the equations into $\frac{a(r^{2021} - 1)}{r - 1} = 200$, $\frac{a(r^{4042} - 1)}{r - 1} = 380$ and $\frac{a(r^{6063} - 1)}{r - 1}$ . Dividing the second equation by the first gives $$\frac{r^{4202} - 1}{r^{2021} - 1} = \frac{380}{200} = \frac{19}{10}$$ Let $x = r^{2021}$. The equation then becomes $\frac{x^2 - 1}{x + 1} = \frac{19}{10}$. This simplifies to $x + 1 = \frac{19}{10}$, so $r^{2021} = x = \frac{9}{10}$. We finish off with the following
\begin{align*}
\frac{a(r^{2021} - 1)}{r - 1} &= 200\\
\frac{a(r^{2021} - 1)}{r - 1}\cdot \frac{r^{6063} - 1}{r^{2021} - 1} &= 200\cdot \frac{r^{6063} - 1}{r^{2021} - 1}\\
\frac{a(r^{6063} - 1)}{r - 1} &= 200 \cdot \frac{x^3 - 1}{x - 1}\\
\frac{a(r^{6063} - 1)}{r - 1} &= 200 \cdot \frac{(9/10)^3 - 1}{(9/10) - 1}\\
\frac{a(r^{6063} - 1)}{r - 1} &= 542
\end{align*}
\textbf{Solution 2}\newline

This solution does not use factorisations, and is the one I originally came up with.
\begin{align*}
380 &= a +ar + \dotsb +ar^{4041}\\
380 &= (a +ar +\dotsb +ar^{2020}) + (ar^{2021} + ar^{2022} + \dotsb + ar^{4041})\\
380 &= (200) + r^{2021}(a + ar + \dotsb + ar^{2020})\\
380 &= (200) + r^{2021}(200)\\
\end{align*}
So $r^{2021} = \frac{9}{10}$. We apply the same technique.
\begin{align*}
a +ar + \dotsb +ar^{6062} &= a + ar + \dotsb +ar^{6062}\\
&= (a +ar +\dotsb +ar^{4041}) + (ar^{4042} + ar{4043} +  \dotsb + ar^{6062})\\
&= (380) + r^{4042}(a + ar + \dotsb + ar^{2020})\\
&= (380) + \left(\frac{9}{10}\right)^2\cdot(200)\\
&= 542
\end{align*}
  
  \item % Jon
  {\itshape Isaac is planning a nine-day holiday. Every day he will go surfing, or water skiing, or he will rest. On any given day he does just one of these three things. He never does different water-sports on consecutive days. How many schedules are possible for the holiday?}
  
  Let $a_n$ be the amount of schedules possible in an $n$ day holiday. We wish to find $a_9$. Let $s_n$ be the amount of schedules in an $n$ day holiday where Isaac's first activity must be surfing, simirlarly let $w_n$ be the amount of schedules in an $n$ day holiday that starts with water skiing. We will find a recurrence relation between these sequences.

Consider a schedule that starts with swimming. If you swap all the swimming days with skiing days, and vice versa, you get a valid schedule that starts with water skiing. This shows that $s_n = w_n$.

We find a recurrence for $w_n$ by considering a schedule $n$ days long. By definition the schedule must start with water skiing. The activity after that could either be water skiing again, or a rest day. There are $w_{n - 1}$ ways of having a schedule with water skiing twice in a row, and if a rest day is taken the other $n - 2$ days can be filled freely, which gives $a_{n - 2}$ ways. Thus $$w_n = w_{n - 1} + a_{n - 2}\text{ for } n > 2$$

We find a recurrence for $a_n$ by considering a schedule $n$ days long. The day can either start with water skiing or swimming, there are $w_n$ and $s_n$ ways of doing this respectively. If a rest day is taken, the other $n - 1$ days can be filled in freely, giving $a_{n - 1}$ more ways of filling the schedule. Thus $a_n = w_n + s_n + a_{n - 1}$, but recall that $s_n = w_n$ so $$a_n = 2w_n + a_{n - 1} \text{ for } n > 1$$

It is easy to check that $a_1 = 3, w_1 = 1, a_2 = 7, w_2 = 2$. We do the rest by table
\begin{table}[h]
	\centering
	\begin{tabular}{c | c | c}
	n& $a_n$ & $b_n$\\
	\hline
1 & 3 & 1 \\
2 & 7 & 2 \\
3 & 17 & 5 \\
4 & 41 & 12 \\
5 & 99 & 29 \\
6 & 239 & 70 \\
7 & 577 & 169 \\
8 & 1393 & 408 \\
9 & 3363 & 985 \\
	\toprule
	\end{tabular}
\end{table}

Which gives us $a_9 = 3363$
  
  
  \item % Danielle from Ukrainian MO
  {\itshape A positive integer $N$ has exactly $2021$ positive divisors (including $1$ and $N$ itself), and it is divisible by $2021$.
  Prove that $N$ is not divisible by $2021^{43}$.}

  We recall that if
  \[
    N = {p_1}^{a_1} {p_2}^{a_2} \cdots {p_k}^{a_k}
  \]
  is the prime factorisation of $N$, then the number of divisors of $N$ is given by
  \[
    (a_1 + 1)(a_2 + 1) \cdots (a_k + 1).
  \]

  We thus investigate the solutions to
  \[
    (a_1 + 1)(a_2 + 1) \cdots (a_k + 1) = 2021.
  \]
  We know that $2021 = 43 \times 47$, which are both prime, and so the only ways of factorising $2021$ as a product of some number of integers are $2021$ and $43 \times 47$.

  Since $N$ is divisible by $2021$, $N$ has at least the two primes factors $43$ and $47$, and so it is the second factorisation that is relevant: we must have that
  \begin{align*}
    a_1 + 1 & = 43 & \text{ and } && a_2 + 1 & = 47
  \end{align*}
  or vice versa. Thus the only options for $N$ are $43^{42} \times 47^{46}$, or $43^{46} \times 47^{42}$, neither of which is divisible by $2021^{43} = 43^{43} \times 47^{43}$.
    
  
  \item % Ralph: found in one of the books. Can't remember which one.
  {\itshape Let $a$, $b$, $c$, $x$, $y$ and $z$ be positive real numbers with $a + b + c = x + y + z$.
  Prove that 
  \[ \frac{a}{x + y} + \frac{b}{y + z} + \frac{c}{z + x} + \frac{x}{a + b} + \frac{y}{b + c} + \frac{z}{c + a} > 2. \]}
  
  Increasing the value of each of the denominators decreases the value of each fraction, and so
  \begin{align*}
    \frac{a}{x + y} & + \frac{b}{y + z} + \frac{c}{z + x} + \frac{x}{a + b} + \frac{y}{b + c} + \frac{z}{c + a} \\
    & > \frac{a}{x + y + z} + \frac{b}{y + z + x} + \frac{c}{z + x + y} + \frac{x}{a + b + c} + \frac{y}{b + c + a} + \frac{z}{c + a + b} \\
    & = \frac{a + b + c}{x + y + z} + \frac{x + y + z}{a + b + c} \\
    & = 2.
  \end{align*}
  
  \end{enumerate}
  
  \vfill
  % ASCII art
  \centering
  \begin{BVerbatim}
    ,,,,,
    (o   o)
     /. .\ 
    (_____)
      : :
     ##O##
   ,,,: :,,,
  _)\ : : /(____
  {  \     /  ___}
  \/)     ((/
   (_______)
     :   :
     :   :
    / \ / \
  \end{BVerbatim}
  \end{document}
