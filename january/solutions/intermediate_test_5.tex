\documentclass{article}

\usepackage{mathtools,amsfonts,amssymb}
\usepackage{enumerate}
\usepackage{fullpage}
\usepackage{fancyvrb}
\usepackage{hyperref}


\begin{document}
\thispagestyle{empty}

\begin{center}
  \textbf{\Large Intermediate Test 5}
  \\ \vspace{1em}
  \textbf{\large January Camp 2021}
  \\ \vspace{1em}
  \textbf{\large Time: $4$ hours}
\end{center}

\vspace{12pt}

\begin{enumerate}[1.]

\item % Tim
{\itshape In the acute-angled triangle $ABC$, the foot of the perpendicular from $B$ to $CA$ is $E$. Let $l$ be the tangent to the circumcircle of $\triangle ABC$ at $B$. The foot of the perpendicular from $C$ to $l$ is $F$. Prove that $EF$ is parallel to $AB$.}

We notice that $\angle CEB + \angle CFB = 90^\circ + 90^\circ = 180^\circ$, which implies $CEBF$ is cyclic. We have $\angle CAB = \angle CBF$ by tan-cord theorem, and $\angle CBF = \angle CEF$ by $CEBF$ cyclic. This gives $\angle CAB = \angle CEF$, so $EF||AB$ by corresponding angles.

\item % Russian, probably. Put in by Taariq
{\itshape Let $x$ and $y$ be distinct real numbers such that $$x + 4 = (y - 2)^2 \qquad\text{and}\qquad y + 4 = (x - 2)^2.$$
Find $x^2 + y^2$.}

We take a difference of squares.
\begin{align*}
(x - 2)^2 - (y - 2)^2 &= (y + 4) - (x + 4)\\
(x - y)(x + y - 4) &= y - x\\
(x + y - 4) &= -1\\
x + y &= 3\\
\end{align*}
From line 2 to line 3, we can divide by $(x - y)$ because $x - y \neq 0$ by the fact that $x$ and $y$ are distinct.

Notice our starting equations can be turned into $$y^2 = x + 4y \quad \text{and}\quad x^2 = 4x + y$$ Adding those equations together we get
\begin{align*}
x^2 + y^2 &= 5x + 5y\\
&= 5(x + y)\\
&=15
\end{align*}

\item % Tim
{\itshape There is a group of 6 people that are very suspicious of each other. It is very important to these people which of the rest of the group consider them to be friends. Now, because of the nature of this group, friendship is not a two way street! Person A may consider himself friends with Person B while Person B does not consider the same about Person A. A true friendship is a pair of people who both consider each other to be friends. If we are told that each of these 6 people considers exactly 3 others to be their friend, what is the minimum and maximum number of true friendships that can exist among these 6 people?}


\item % Ireland 2017 Q7
{\itshape Consider relatively prime integers $a$ and $b$ and a prime number $p$.
Show that
\[ \gcd(ab, a^2+pb^2) = \begin{cases*} 1 & if $p \nmid a$ \\ p & if $p \mid a$ \end{cases*}. \]}

Case 1: $p \nmid a$ \par
We will assume for contradiction that $\gcd(ab, a^2+pb^2) > 1$. Let $d = \gcd(ab, a^2+pb^2)$ and let $q$ be any prime divisor of $d$. 
Then we have that $d \mid ab$ and $d \mid a^2+pb^2 \implies q \mid ab$ and $q \mid a^2+pb^2$. Since $q \mid ab$ and $a$ and $b$ are relatively prime, we have that either $q \mid a$ or $q \mid b$. \par
If $q \mid a$: since we had $q \mid a^2 + pb^2 \implies q \mid pb^2 \implies q \mid p$ or $q \mid b$. The first case would mean that $q = p$ since both are prime and thus that $p \mid a$ which would be a contradiction and the second would contradict the condition that $a$ and $b$ cannot both be divisible by $q$. \par
If $q \mid b$: $q \mid a^2 + pb^2 \implies q \mid a^2 \implies q \mid a$ which is a contradiction since we assumed only one of $a$ or $b$ is divisible by $q$. \par
In conclusion, we have that $\gcd(ab, a^2+pb^2) > 1$ is not true and thus $\gcd(ab, a^2+pb^2) = 1.$ \newline

Case 2: $p \mid a$ \par
Let $a = pk$ where $k$ is a natural number. Note that since $\gcd(a, b) = 1$ we will also have $\gcd(k, b) = 1$ and $\gcd(p, b) = 1.$ \par
We can then say $\gcd(ab, a^2+pb^2) = \gcd(pkb, p^2k^2 + pb^2) = p\cdot \gcd(kb, pk^2 + b^2).$ Assume for contradiction that $\gcd(kb, pk^2 + b^2) > 1$. Let $d = \gcd(kb, pk^2 + b^2)$ and let $q$ be any prime divisor of $d$. Then either $q \mid k$ or $q \mid b$. \par
If $q \mid k$: since we had $q \mid pk^2 + b^2 \implies q \mid b^2 \implies q \mid b$. However, this would be a contradiction since we assumed that $k$ and $b$ cannot both be divisible by $q$. \par
If $q \mid b$ we will have  $q \mid pk^2 + b^2 \implies q \mid pk^2 \implies q \mid p$ or $q \mid k$ . However, since we had that $\gcd(k, b) = 1$ and $\gcd(p, b) = 1$, we cannot have $k$ or $p$ divisible by $q$ as this would mean that $\gcd(k, b) > 1$ or $\gcd(p, b) > 1$. \par
In conclusion we will have that $\gcd(kb, pk^2 + b^2) = 1 \implies \gcd(ab, a^2+pb^2) = p\cdot 1 = p$.


\item % Tim
{\itshape A square-based pyramid has all of its edges the same length.
A cube is placed inside so that one of its faces lies on the base of the pyramid, and the opposite face has an edge along each side of the pyramid (think -- the natural way to put a cube in a pyramid).
Show that the sum of any of the cube's edge lengths with any of the cube's face diagonal lengths is the same as the edge length of the pyramid.}


\item % Ireland 2017 Q15
{\itshape There is a soccer competition with five teams where each team plays exactly one match against each other team.
If one team wins against another team, that team gets $5$ points whereas the losing team gets $0$ points.
If two teams draw, they each get $1$ point if neither team scored a goal and they each get $2$ points if they scored at least one goal.
At the end of the competition it is found that the total points for the teams form five consecutive nonnegative integers.
What is the minimum number of goals scored in this competition?}

\end{enumerate}


\clearpage
~
\vfill
% ASCII art
\centering
% \tiny
The duck in this paper was an affront to God, and so has been removed.
\vfill

\end{document}
