\documentclass{article}

\usepackage{mathtools,amsfonts}
\usepackage{enumerate}
\usepackage{fullpage}
\usepackage{fancyvrb}
\usepackage{hyperref}


\begin{document}
\thispagestyle{empty}

\begin{center}
  \textbf{\Large Advanced Test 5 Solutions}
  % LEVEL is Senior, Intermediate or Beginner
  % NUMBER is the test number: 1, 2, etc.
  \\ \vspace{1em}
  \textbf{\large January Camp 2021}
\end{center}

\vspace{12pt}

\begin{enumerate}[1.]

\item % Tim
{\itshape A square-based pyramid has all of its edges the same length.
A cube is placed inside so that one of its faces lies on the base of the pyramid, and the opposite face has an edge along each side of the pyramid (think -- the natural way to put a cube in a pyramid).
Show that the sum of any of the cube's edge lengths with any of the cube's face diagonal lengths is the same as the edge length of the pyramid.}

Let the base of the square pyramid consist of the square $ABCD$, and the top of the pyramid be point $T$.
Consider the plane $P$ through points $A$, $T$, and $C$; by the way the cube is placed inside the pyramid, this plane passes through four points of the cube, the intersection of the plane with the surface of the cube being diagonals of the upper and lower faces and two opposite edges; let these edges be $WX$ and $YZ$ respectively.
Finally, let the edge lengths of the cube and pyramid be $a$ and $b$ respectively; by Pythagoras, $XY = \sqrt{2} YZ = \sqrt{2} a$ and $AC = \sqrt{2} AT = \sqrt{2} b$.
Considering the intersection of the plane $P$ with the pyramid and the cube, we have a rectangle with $\sqrt{2}$ the ratio of width to height inside an isosceles triangle with a right angle at the top.

We place this on the Cartesian plane centred at the midpoint of the base of the triangle, with the top at $(0,b/\sqrt{2})$ and the other two vertices at $(\pm b/\sqrt{2},0)$; then the vertices of the rectangle are at $(\pm a/\sqrt{2},0)$ and $(\pm a/\sqrt{2},a)$.
The equation of the side of the triangle on the right is $y = b/\sqrt{2} -x$; since the point $(\pm a/\sqrt{2},a)$ of the rectangle lies on this side, we have
\[ a = b/\sqrt{2} -a/\sqrt{2} \iff b = \sqrt{2} a + a \]
as desired.


\item % Ireland 2017 Q15
{\itshape There is a soccer competition with five teams where each team plays exactly one match against each other team.
If one team wins against another team, that team gets $5$ points whereas the losing team gets $0$ points.
If two teams draw, they each get $1$ point if neither team scored a goal and they each get $2$ points if they scored at least one goal.
At the end of the competition it is found that the total points for the teams form five consecutive nonnegative integers.
What is the minimum number of goals scored in this competition?}


\textbf{Solution 1}\\
Ten matches are played each one contributing either 2, 4 or 5 points. Hence the total number of points is between 20 and 50. If the team scores are five consecutive integers, then the total number of points must be a multiple of 5. If the total number of points is 20, all teams will score 4 and if the total number of points is 50 all team totals will be multiples of 5. Neither of these possibilities satisfy the conditions. Therefore, we need to consider the following five cases:
\begin{itemize}
	\item [(a)] scores are 3, 4, 5, 6, 7,
	\item [(b)] scores are 4, 5, 6, 7, 8,
	\item [(c)] scores are 5, 6, 7, 8, 9,
	\item [(d)] scores are 6, 7, 8, 9, 10 and
	\item [(e)] scores are 7, 8, 9, 10, 11.
\end{itemize}
Case (a): Number of wins is odd. 3 wins yield 15 points and the other seven matches yield more than 10 points. The only possibility is one win, 8 0-0 draws and one score draw. But the team that wins must gain at least 3 points in the other matches. Hence this case is impossible.\\
Case (b): The number of wins is even and cannot be 4 as only three teams have a score of five or more and none have a score of 10. No wins means there are 5 score draws and 5 no score draws. The teams scoring 8 and 7 must be involved in 7 score draws to achieve these totals. Hence, the only possibility is 2 wins 2 score draws and 6 no score draws. The table

\begin{table}[ht!]
	\begin{center}
	  \begin{tabular}{ | c | c | c | c | c | c | c |}
		  \hline
		    & A & B & C & D & E & Total\\
		  \hline  
		  A &   & 5 & 1 & 1 & 1 & 8\\
		  \hline  
		  B & 0 &   & 1 & 1 & 5 & 7\\
		  \hline  
		  C & 1 & 1 &   & 2 & 2 & 6\\
		  \hline  
		  D & 1 & 1 & 2 &   & 1 & 5\\
		  \hline  
		  E & 1 & 0 & 2 & 1 &   & 4\\
		  \hline  
	  \end{tabular}
	\end{center}
\end{table}

realizes this possibility. The minimum number of goals in this case is 6.\\
Case (c): The number of wins is odd. No team has more than one win. If the number of wins is five, each team must win one match and all other matches are no score draws. A total of 9 is now impossible. If the number of wins is 3, there most be 3 score draws and 4 no score draws. At least 9 goals are scored in this scenario. If the number of wins is 1, there are 6 score draws and three no score draws. This gives at least 13 goals.\\
Case (d): Again we can calculate the number of wins $(W)$, score draws $(S)$ and no score draws $(N)$ yielding 40 points. The four possibilities are $(W, S, N) = (6, 1 3)$ or (4, 4, 2) or (2, 7, 1) or (0, 10, 0). The minimum number of goals is $W + 2S$ which in each case is bigger than 6.\\
Case (e): Calculating possible values of $(W, S, N)$ we obtain (7, 2, 1), (5, 5, 0) giving more than 6 goals.
Thus the minimum number of goals scored in the tournament is 6.\\
\newline
\textbf{Solution 2}
If the five consecutive scores of the teams are $a-2, a-1, a, a+1, a+2$, the total number of points is $5a$. Each match contributes 5, 4 or 2 points to this total.\\
Let $w$ be the number of matches that did not end in a draw, $d_0$ the number of 0-0 draws and $d_1$ the number of draws with goals. The minimal possible number of goals scored is $g = w + 2d_1$. The number of matches is ${5 \choose 2} = 10$, so we obtain
\begin{eqnarray*}
w + d_0 + d_1 & = & 10\\
5w + 2d_0 + 4d_1 & = & 5a.
\end{eqnarray*}
Eliminating $d_0$ from these equations we obtain $3w + 2d_1 = 5(a-4)$. This equation implies $3w = -2d_1$ (mod 5), hence $w = d_1$ (mod 5).\\
None of $w$, $d_1$ and $w+d_1$ can exceed 10. Hence, $d_1 = w + 5t$ with $-2 \leq t \leq 2$. We obtain $g = w + 2d_1 = 3w + 10t$.
\begin{itemize}
\item  If $t = -2$, we have $w = d_1 + 10 \leq 10$, hence $w = 10$, $d_1 = 0$ and $g = 10$.
\item  If $t = -1$, we have $d_1 = w - 5 \geq 0$ and $w + d_1 = 2w - 5 \leq 10$, hence $5 \leq w + 7$. If $w \geq 6$, we have $g = 3w - 10 \geq 8$. For $w = 5$, we get $g = 5$.
\item  If $t = 0$,  and $w \geq 3$ we have $g = 3w + 10t \geq 9$.
\item  If $t \geq 1$, we have $g \geq 10$.
\end{itemize}
Hence, there are only four cases in which $g \leq 6$. They are all shown in the table below. The values of $d_0$, $a$ and $g$ are obtained from the equations above.

\begin{table}[ht!]
	\begin{center}
	  \begin{tabular}{ | c | c | c | c | c | c |}
		  \hline
		  Case &  $w$ & $d_1$ & $d_0$ & $a$ & $g$ \\
		  \hline  
		  1 & 0 & 0 & 10 & 4 & 0 \\
		  \hline  
		  2 & 1 & 1 &  8 & 5 & 3 \\
		  \hline  
		  3 & 5 & 0 &  5 & 7 & 5 \\
		  \hline
		  4 & 2 & 2 &  6 & 6 & 6\\
		  \hline  
	  \end{tabular}
	\end{center}
\end{table}

Case 1: If $d_0 = 10$ all matches were no score draws, hence all teams scored 5 points contradicting the conditions.\\
Case 2: If $a = 5$, the winning team achieved 7 points. But, if $w = 1$ the team who has one one match cannot have lost any of their matches, hence would have scored at least 8 points.\\
Case 3: If $a = 7$, the maximum number of points a team scored is 9, hence no team won two matches. As $d_1 = 0$, the top team can have scored at most 3 points from the matches it didn't win, hence cannot have got 9 points in total.\\
Case 4: This case is actually possible, as the following table shows.
\begin{table}[ht!]
	\begin{center}
	  \begin{tabular}{ | c | c | c | c | c | c | c |}
		  \hline
		    & A & B & C & D & E & Total\\
		  \hline  
		  A &   & 5 & 1 & 1 & 1 & 8\\
		  \hline  
		  B & 0 &   & 1 & 1 & 5 & 7\\
		  \hline  
		  C & 1 & 1 &   & 2 & 2 & 6\\
		  \hline  
		  D & 1 & 1 & 2 &   & 1 & 5\\
		  \hline  
		  E & 1 & 0 & 2 & 1 &   & 4\\
		  \hline  
	  \end{tabular}
	\end{center}
\end{table}

Therefore, the minimum number of goals scored is 6.


\item % Norwegian MO 1998, P1
{\itshape Given a sequence $a_0$, $a_1$, $a_2$, $\cdots$, of natural numbers with $a_0 = 1$ and $a_{n}^2 > a_{n - 1}a_{n + 1}$ for all $n > 0$, show that 
\begin{enumerate}[a)]
  \item $a_n < a_1^n$ for all $n > 1$.
  \item $a_n > n$ for all $n$.
\end{enumerate}}

\begin{enumerate}[a)]
  \item We shall first prove that $a_n < a_1 \cdot a_{n - 1}$. We shall do this by induction.
  \begin{itemize}
    \item Induction Hypothesis: $a_n < a_1 \cdot a_{n - 1}$ for all $n > 1$.
    \item Base Case: $n = 2$. We have $a_{1}^2 > a_{1 - 1}a_{1 + 1} = a_0a_2 = a_2$. Therefore $a_2 < a_1 \cdot a_{2 - 1}$ so the base case holds.
    \item Assume true for $n = k$.
    \item Prove true for $n = k + 1$. We have that $a_k < a_1 \cdot a_{k - 1} \implies a_k \cdot a_{k + 1} < a_1 \cdot a_{k - 1} \cdot a_{k + 1} < a_1 \cdot a_k^2$. Simplifying this gives $a_{k + 1} < a_1 \cdot a_k$ which is exactly what we wanted to show in the inductive step.
  \end{itemize}
  Hence, by the Principle of Mathematical Induction, we have $a_n < a_1 \cdot a_{n - 1}$ for all $n > 1$. Now we simply note that this gives 
  $$a_n < a_1 \cdot a_{n - 1} < a_1 \cdot a_1 \cdot a_{n - 2} < \cdot < a_1^n \cdot a_0 = a_1^n.$$
  \item We shall prove that $a_n > a_{n - 1}$. Suppose for contradiction that there exists some $k$ such that $a_k \le a_{k - 1}$. From this we get $a_k \cdot a_{k + 1} \le a_{k - 1} \cdot a_{k + 1} < a_k^2 \implies a_{k + 1} < a_k$. Now one can prove by induction that if there exists some $k$ such that $a_k \le a_{k - 1}$, then the rest of the sequence must be strictly decreasing. However, since the sequence is defined on positive integers, we eventually cannot decrease further. Hence we have a contradiction, so there cannot be a $k$ such that $a_k \le a_{k - 1}$. Hence $a_n > a_{n - 1}$ for all $n$. So we must then have that $a_n > a_1 + {n - 1} = n$.
\end{enumerate}


\item % ELMO 2013 Shortlist, C8
{\itshape There are $20$ people at a party. Each person holds some number of coins. Every minute, each person who has at least $19$ coins simultaneously gives one coin to every other person at the party. What is the smallest number of coins that could be at the party such that at least one coin is given out in every minute?}

Notice that the construction $(0, 1, \cdots, 19)$ satisfies the condition that the coin sharing never stops. It uses $190$ coins. We shall prove that $190$ is indeed the minimum.

Suppose person $i$ hands out some of their coins. They hand one coin to each person, so we can label each coin that was handed out as $(i, j)$ to denote that person $i$ gave that coin to person $j$. Now, if person $j$ ever hands out coins, we shall say that they must give person $i$ the coin $(i, j)$ back to person $j$. Such a coin must exist since the assumption is $i$ gave a coin to $j$ and $j$ has not handed out coins since. If the coin sharing does not stop, then each person must eventually hand out coins themselves. Thus we must eventually have that every coin with pair $(i, j)$ and $i \ne j$ must be present at the party. Thus, the lower bound on the number of coins is 
$$\binom{20}{2} = 190$$
Since we have a construction for when it is exactly $190$, we have found the minimum.


\item % 
{\itshape Given a function $f: \mathbb{Q} \rightarrow \mathbb{Q}$ satisfying
\begin{enumerate}[i)]
  \item $f(0) = 2$ and $f(1) = 3$
  \item For all rational numbers $x$ and all integers $n$, we have 
  $$f(x + n) - f(x) = n(f(x + 1) - f(x)) $$
  \item For all non-zero rational numbers $x$, we have $f(x) = f(\frac{1}{x})$
\end{enumerate}
Find all possible rational values of $x$ such that $f(x) = 2021$.}

By taking $n = 2$ in condition ii, we find that $f(x + 2) = 2f(x + 1) - f(x)$ for all rational numbers $x$. 

Taking $x = 0$ in condition ii, we have that $f(n) = f(0) + n(f(1) - f(0)) = 2 + n$ for all integers $n$.

We now show by strong induction on the denominator $q$ that for all natural numbers $q$ and all integers $p$ with $\gcd(p, q) = 1$, we have that $f\left(\frac{p}{q}\right) = pq + 2$. This is true for $q = 1$ since we have already seen that $f(n) = n + 2$ for all integers $n$. Suppose that for all $0 < k < q$, we have that $f\left(\frac{n}{k}\right) = nk + 2$ whenever $n$ and $k$ are relatively prime. We will show that the same is true of $q$.

First, suppose that $\gcd(n, q) = 1$, and $0 < n < q$. Then by the induction hypothesis, we have that $f\left(\frac{q}{n}\right) = qn + 2$, and so by condition iii, we have that $f\left(\frac{n}{q}\right) = qn + 2$ as well.

We now show that for $0 < n < q$ such that $\gcd(n, q) = 1$, we have that $f\left(\frac{n}{q} + 1\right) = q^2 + nq + 2$. Indeed,
\begin{align*}
  f\left(\frac{n}{q} + 1\right) & = 2f\left(\frac{n}{q}\right) - f\left(\frac{n}{q} - 1\right) \\
  & = 2(nq + 2) - f\left(-\frac{q - n}{q}\right) \\
  & = 2(nq + 2) - f\left(\frac{-q}{q - n}\right) \\
  & = 2(nq + 2) - (-q^2 + nq + 2) = q^2 + nq + 2,
\end{align*}
where the antepenultimate equality follows from condition iii, and the penultimate equality follows by the induction hypothesis since $q - n$ is positive and smaller than $q$.

It follows by condition ii that
\begin{align*}
  f\left(\frac{mq + n}{q}\right) & = f\left(\frac{n}{q} + m\right) \\
  & = f\left(\frac{n}{q}\right) + m\left(f\left(\frac{n}{q} + 1\right) - f\left(\frac{n}{q}\right)\right) \\
  & = qn + 2 + m(q^2 + qn + 2 - qn - 2) = q(mq + n) + 2
\end{align*}
for all integers $m$ and all positive integers $n$ such that $n < q$ and $\gcd(n, q) = 1$.

Since every integer $k$ that is relatively prime to $q$ can be written as $k = mq + n$ where $0 < n < q$ and $\gcd(n, q) = 1$, we see that our claim holds for $q$ as well. Thus the claim holds for every positive integer $q$ by the principle of strong mathematical induction.

Finally, if $\frac{p}{q}$ is a rational number in its lowest terms (where we may assume that $q$ is positive), we have that
\[
  f\left(\frac{p}{q}\right) = 2021 \iff pq + 2 = 2021 \iff pq = 2019.
\]
The factors of $2019$ are $1, 3, 673$, and $2019$, so we see that the complete set of solutions is
\[
  \left\{ \frac{1}{2019}, \frac{3}{673}, \frac{673}{3}, 2019 \right\}.
\]


\item % Turkic TST 2008, Q1
{\itshape Let $\triangle ABC$ have $ \angle B>\angle C$. The internal angle bisector of $A$ intersects $BC$ at $D$ and the external angle bisector of $A$ intersect $BC$ at $E$. $ P$ is a variable point on $ EA$ such that $ A$ is on $EP$. $ DP$ intersects $ AC$ at $ M$ and $ ME$ intersects $ AD$ at $ Q$. Prove that all lines $ PQ$ intersect at a unique point as $ P$ changes.}

Let $PQ$ and $BC$ intersect in point $N$.\\
Menelaus applied to triangle $AEC$ and collinear points $P,M,D$ yields: $$\frac{PA}{PE} \cdot \frac{ED}{DC} \cdot \frac{CM}{MA} = 1.$$
Menelaus applied to triangle $ADC$ and collinear points $E,Q,M$ yields: $$\frac{ED}{EC} \cdot \frac{CM}{MA} \cdot \frac{AQ}{QD} = 1.$$
Side by side division gives: $$\frac{PA}{PE} \cdot \frac{EC}{DC} \cdot \frac{QD}{AQ} = 1.$$
Finally, Menalaus applied to triangle $AED$ and collinear points $P,Q,N$ yields: $$\frac{PA}{PE} \cdot \frac{EN}{ND} \cdot \frac{QD}{AQ} = 1.$$
Therefore, $$\frac{EN}{ND} = \frac{EC}{DC} = \frac{EB}{BD}$$ and hence $N = B$.\\
(The last equality followed from $\frac{EB}{EC} = \frac{AB}{AC} = \frac{BD}{DC}$.)


\end{enumerate}

\clearpage
~
\vfill
% ASCII art
\centering
% \tiny
\begin{BVerbatim}
                                                      ____
                                                     /    `.
                                                    /-----.|          ____
                                                ___/___.---`--.__.---'    `--.
                                  _______.-----'           __.--'             )
                              ,--'---.______________..----'(  __         __.-'
                                        `---.___,-.|(a (a) /-'  )___.---'
                                                `-.>------<__.-'
            ______                       _____..--'      //
    __.----'      `---._                `._.--._______.-'/))
,--'---.__              -_                  _.-(`-.____.'// \
          `-._            `---.________.---'    >\      /<   \
              \_             `--.___            \ \-__-/ /    \
                \_                  `----._______\ \  / /__    \
                  \                      /  |,-------'-'\  `-.__\
                   \                    (   ||            \      )
                    `\                   \  ||            /\    /
                      \                   >-||  @)    @) /\    /
                      \                  ((_||           \ \_.'|
                       \                    ||            `-'  |
                       \                    ||             /   |
                        \                   ||            (   '|
                        \                   ||  @)     @)  \   |
                         \                  ||              \  )
                          `\_               `|__         ____\ |
                             \_               | ``----'''     \|
                               \_              \    .--___    |)
                                 `-.__          \   |     \   |
                                      `----.___  \^/|      \/\|
                                               `--\ \-._  / | |   (FL)
                                                   \ \  `'  \ \
                                            __...--'  )     (  `-._
                                           (_        /       `.    `-.__
                                             `--.__.'          `.       )
                                                                 `.__.-'
\end{BVerbatim}
\vfill

\end{document}
