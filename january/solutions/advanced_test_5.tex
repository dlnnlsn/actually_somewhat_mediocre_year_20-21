\documentclass{article}

\usepackage{mathtools,amsfonts}
\usepackage{enumerate}
\usepackage{fullpage}
\usepackage{fancyvrb}
\usepackage{hyperref}


\begin{document}
\thispagestyle{empty}

\begin{center}
  \textbf{\Large Advanced Test 5 Solutions}
  % LEVEL is Senior, Intermediate or Beginner
  % NUMBER is the test number: 1, 2, etc.
  \\ \vspace{1em}
  \textbf{\large January Camp 2021}
\end{center}

\vspace{12pt}

\begin{enumerate}[1.]

\item % Tim
{\itshape A square-based pyramid has all of its edges the same length.
A cube is placed inside so that one of its faces lies on the base of the pyramid, and the opposite face has an edge along each side of the pyramid (think -- the natural way to put a cube in a pyramid).
Show that the sum of any of the cube's edge lengths with any of the cube's face diagonal lengths is the same as the edge length of the pyramid.}


\item % Ireland 2017 Q15
{\itshape There is a soccer competition with five teams where each team plays exactly one match against each other team.
If one team wins against another team, that team gets $5$ points whereas the losing team gets $0$ points.
If two teams draw, they each get $1$ point if neither team scored a goal and they each get $2$ points if they scored at least one goal.
At the end of the competition it is found that the total points for the teams form five consecutive nonnegative integers.
What is the minimum number of goals scored in this competition?}


\item % Norwegian MO 1998, P1
{\itshape Given a sequence $a_0$, $a_1$, $a_2$, $\cdots$, of natural numbers with $a_0 = 1$ and $a_{n}^2 > a_{n - 1}a_{n + 1}$ for all $n > 0$, show that 
\begin{enumerate}[a)]
  \item $a_n < a_1^n$ for all $n > 1$.
  \item $a_n > n$ for all $n$.
\end{enumerate}}

\begin{enumerate}[a)]
  \item We shall first prove that $a_n < a_1 \cdot a_{n - 1}$. We shall do this by induction.
  \begin{itemize}
    \item Induction Hypothesis: $a_n < a_1 \cdot a_{n - 1}$ for all $n > 1$.
    \item Base Case: $n = 2$. We have $a_{1}^2 > a_{1 - 1}a_{1 + 1} = a_0a_2 = a_2$. Therefore $a_2 < a_1 \cdot a_{2 - 1}$ so the base case holds.
    \item Assume true for $n = k$.
    \item Prove true for $n = k + 1$. We have that $a_k < a_1 \cdot a_{k - 1} \implies a_k \cdot a_{k + 1} < a_1 \cdot a_{k - 1} \cdot a_{k + 1} < a_1 \cdot a_k^2$. Simplifying this gives $a_{k + 1} < a_1 \cdot a_k$ which is exactly what we wanted to show in the inductive step.
  \end{itemize}
  Hence, by the Principle of Mathematical Induction, we have $a_n < a_1 \cdot a_{n - 1}$ for all $n > 1$. Now we simply note that this gives 
  $$a_n < a_1 \cdot a_{n - 1} < a_1 \cdot a_1 \cdot a_{n - 2} < \cdot < a_1^n \cdot a_0 = a_1^n.$$
  \item We shall prove that $a_n > a_{n - 1}$. Suppose for contradiction that there exists some $k$ such that $a_k \le a_{k - 1}$. From this we get $a_k \cdot a_{k + 1} \le a_{k - 1} \cdot a_{k + 1} < a_k^2 \implies a_{k + 1} < a_k$. Now one can prove by induction that if there exists some $k$ such that $a_k \le a_{k - 1}$, then the rest of the sequence must be strictly decreasing. However, since the sequence is defined on positive integers, we eventually cannot decrease further. Hence we have a contradiction, so there cannot be a $k$ such that $a_k \le a_{k - 1}$. Hence $a_n > a_{n - 1}$ for all $n$. So we must then have that $a_n > a_1 + {n - 1} = n$.
\end{enumerate}

\item % ELMO 2013 Shortlist, C8
{\itshape There are $20$ people at a party. Each person holds some number of coins. Every minute, each person who has at least $19$ coins simultaneously gives one coin to every other person at the party. What is the smallest number of coins that could be at the party such that at least one coin is given out in every minute?}

Notice that the construction $(0, 1, \cdots, 19)$ satisfies the condition that the coin sharing never stops. It uses $190$ coins. We shall prove that $190$ is indeed the minimum.

Suppose person $i$ hands out some of their coins. They hand one coin to each person, so we can label each coin that was handed out as $(i, j)$ to denote that person $i$ gave that coin to person $j$. Now, if person $j$ ever hands out coins, we shall say that they must give person $i$ the coin $(i, j)$ back to person $j$. Such a coin must exist since the assumption is $i$ gave a coin to $j$ and $j$ has not handed out coins since. If the coin sharing does not stop, then each person must eventually hand out coins themselves. Thus we must eventually have that every coin with pair $(i, j)$ and $i \ne j$ must be present at the party. Thus, the lower bound on the number of coins is 
$$\binom{20}{2} = 190$$
Since we have a construction for when it is exactly $190$, we have found the minimum.

\item % 
{\itshape Given a function $f: \mathbb{Q} \rightarrow \mathbb{Q}$ satisfying
\begin{enumerate}[i)]
  \item $f(0) = 2$ and $f(1) = 3$
  \item For all rational numbers $x$ and all integers $n$, we have 
  $$f(x + n) - f(x) = n(f(x + 1) - f(x)) $$
  \item For all non-zero rational numbers $x$, we have $f(x) = f(\frac{1}{x})$
\end{enumerate}
Find all possible rational values of $x$ such that $f(x) = 2021$.}

By taking $n = 2$ in condition ii, we find that $f(x + 2) = 2f(x + 1) - f(x)$ for all rational numbers $x$. This allows us to prove by induction that $f(n) = n + 2$ for all integers $n$. Indeed, we have that $f(0) = 0 + 2$ and $f(1) = 1 + 2$ by condition i, so the base case is true. Suppose that $f(n) = n + 2$ and $f(n + 1) = n + 3$ for some integer $n$. Then $f(n + 2) = 2f(n + 1) - f(n) = 2(n + 3) - (n + 2) = n + 4$, so the claim is also true for $n + 2$. It follows by mathematical induction that $f(n) = n + 2$ for all non-negative integers $n$. A similar proof works to show that this is true for negative integers as well.

We now show by strong induction on the denominator $q$ that for all natural numbers $q$ and all integers $p$ with $\gcd(p, q) = 1$, we have that $f\left(\frac{p}{q}\right) = pq + 2$. This is true for $q = 1$ since we have already seen that $f(n) = n + 2$ for all integers $n$. Suppose that for all $0 < k < q$, we have that $f\left(\frac{n}{k}\right) = nk + 2$ whenever $n$ and $k$ are relatively prime. We will show that the same is true of $q$.

First, suppose that $\gcd(n, q) = 1$, and $0 < n < q$. Then by the induction hypothesis, we have that $f\left(\frac{q}{n}\right) = qn + 2$, and so by condition iii, we have that $f\left(\frac{n}{q}\right) = qn + 2$ as well.

We now show that for $0 < n < q$ such that $\gcd(n, q) = 1$, we have that $f\left(\frac{n}{q} + 1\right) = q^2 + nq + 2$. Indeed,
\begin{align*}
  f\left(\frac{n}{q} + 1\right) & = 2f\left(\frac{n}{q}\right) - f\left(\frac{n}{q} - 1\right) \\
  & = 2(nq + 2) - f\left(-\frac{q - n}{q}\right) \\
  & = 2(nq + 2) - f\left(\frac{-q}{q - n}\right) \\
  & = 2(nq + 2) - (-q^2 + nq + 2) = q^2 + nq + 2,
\end{align*}
where the antepenultimate equality follows from condition iii, and the penultimate equality follows by the induction hypothesis since $q - n$ is positive and smaller than $q$.

It follows by condition ii that
\begin{align*}
  f\left(\frac{mq + n}{q}\right) & = f\left(\frac{n}{q} + m\right) \\
  & = f\left(\frac{n}{q}\right) + m\left(f\left(\frac{n}{q} + 1\right) - f\left(\frac{n}{q}\right)\right) \\
  & = qn + 2 + m(q^2 + qn + 2 - qn - 2) = q(mq + n) + 2
\end{align*}
for all integers $m$ and all positive integers $n$ such that $n < q$ and $\gcd(n, q) = 1$.

Since every integer $k$ that is relatively prime to $q$ can be written as $k = mq + n$ where $0 < n < q$ and $\gcd(n, q) = 1$, we see that our claim holds for $q$ as well. Thus the claim holds for every positive integer $q$ by the principle of strong mathematical induction.

Finally, if $\frac{p}{q}$ is a rational number in its lowest terms (where we may assume that $q$ is positive), we have that
\[
  f\left(\frac{p}{q}\right) = 2021 \iff pq + 2 = 2021 \iff pq = 2019.
\]
The factors of $2019$ are $1, 3, 673$, and $2019$, so we see that the complete set of solutions is
\[
  \left\{ \frac{1}{2019}, \frac{3}{673}, \frac{673}{3}, 2019 \right\}.
\]


\item % Turkic TST 2008, Q1
{\itshape Let $\triangle ABC$ have $ \angle B>\angle C$. The internal angle bisector of $A$ intersects $BC$ at $D$ and the external angle bisector of $A$ intersect $BC$ at $E$. $ P$ is a variable point on $ EA$ such that $ A$ is on $EP$. $ DP$ intersects $ AC$ at $ M$ and $ ME$ intersects $ AD$ at $ Q$. Prove that all lines $ PQ$ intersect at a unique point as $ P$ changes.}

\end{enumerate}

\clearpage
~
\vfill
% ASCII art
\centering
% \tiny
\begin{BVerbatim}
                                                      ____
                                                     /    `.
                                                    /-----.|          ____
                                                ___/___.---`--.__.---'    `--.
                                  _______.-----'           __.--'             )
                              ,--'---.______________..----'(  __         __.-'
                                        `---.___,-.|(a (a) /-'  )___.---'
                                                `-.>------<__.-'
            ______                       _____..--'      //
    __.----'      `---._                `._.--._______.-'/))
,--'---.__              -_                  _.-(`-.____.'// \
          `-._            `---.________.---'    >\      /<   \
              \_             `--.___            \ \-__-/ /    \
                \_                  `----._______\ \  / /__    \
                  \                      /  |,-------'-'\  `-.__\
                   \                    (   ||            \      )
                    `\                   \  ||            /\    /
                      \                   >-||  @)    @) /\    /
                      \                  ((_||           \ \_.'|
                       \                    ||            `-'  |
                       \                    ||             /   |
                        \                   ||            (   '|
                        \                   ||  @)     @)  \   |
                         \                  ||              \  )
                          `\_               `|__         ____\ |
                             \_               | ``----'''     \|
                               \_              \    .--___    |)
                                 `-.__          \   |     \   |
                                      `----.___  \^/|      \/\|
                                               `--\ \-._  / | |   (FL)
                                                   \ \  `'  \ \
                                            __...--'  )     (  `-._
                                           (_        /       `.    `-.__
                                             `--.__.'          `.       )
                                                                 `.__.-'
\end{BVerbatim}
\vfill

\end{document}
