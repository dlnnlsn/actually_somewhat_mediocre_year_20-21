\documentclass{article}

\usepackage{mathtools,amsfonts}
\usepackage{enumerate}
\usepackage{fullpage}
\usepackage{fancyvrb}
\usepackage{hyperref}
\usepackage{booktabs}

\begin{document}
\thispagestyle{empty}

\begin{center}
  \textbf{\Large Intermediate Test 4 Solutions}
  % LEVEL is Senior, Intermediate or Beginner
  % NUMBER is the test number: 1, 2, etc.
  \\ \vspace{1em}
  \textbf{\large January Camp 2021}
  \\ \vspace{1em}
  \textbf{\large Time: $2\frac{1}{2}$ hours}
\end{center}

\vspace{24pt}

\begin{enumerate}[1.]

\item % Tim
{\itshape Find all functions $f : \mathbb{R} \to \mathbb{R}$ such that for all $x, y \in \mathbb{R}$ we have that
\[ xf(y) = yf(x). \]}

Let $y = 1$ $$xf(1) = f(x)$$ If we let $m = f(1)$, we see that the solution is just $f(x) = mx$ for $m \in \mathbb{R}$.

Check for $f(x) = mx$:$$LHS = xf(y) = xmy = mxy$$ $$RHS = yf(x) = ymx = mxy$$ $LHS = RHS$, so the function checks.

\item % Croatia 2017 2nd round 3.2
{\itshape Find all positive integers $m$ such that $2^{m^2}-4$ is divisible by $7$.}

We want to find $m$ such that $2^{m^2}\equiv _{7} 4$. 
The possible remainders when $2^k$ is divided by 7 where $k$ is a natural number will be as follows:
\begin{align*} 
2^1 = 2 \equiv _{7} 2\\
2^2 = 4 \equiv _{7} 4\\
2^3 = 8 \equiv _{7} 1\\
2^4 = 16 \equiv _{7} 2\\
2^5 = 32 \equiv _{7} 4\\
2^6 = 64 \equiv _{7} 1
\end{align*}
This pattern will continue and we will see that $2^k$ mod 7 will be one of 1, 2, or 4. 
We also note that $2^k \equiv _{7} 4$ when $k \equiv _{3} 2$. 
If $k = m^2$ we have that $m^2 \equiv _{3} 2$. However, a perfect square cannot leave a remainder of 2 when divided by 3 which means that there will be no solutions for m.


\item % Croatia 2017 2nd round 2.4
{\itshape Consider a triangle $ABC$ with circumcentre $O$.
The angle bisector of $\angle BAC$ meets the opposite side $BC$ at $D$, and the altitude from $B$ onto $AD$ intersects line $AO$ at $E$.
Show that $A$, $B$, $D$, and $E$ are concyclic.}


\item % Tim
{\itshape Consider a $3\times3\times3$ 3-dimensional chess cube with some hyperrooks.
Hyperrooks can move along any direction parallel to an edge of the cube (like a normal rook, but also up and down).
What is the maximum number of hyperrooks you can place in the chess cube without any of them attacking each other?}

First, we prove that 9 hyperrooks is the maximum amount of hyperrooks you can place on a chess cube. Consider the 9 columns (in the up-down direction) of the chess cube. If a column contains 2 hyperrooks, the hyperrooks would attack each other. Therefore, each column has at most 1 hyperrook. Since there are 9 columns, there are at most 9 hyperrooks. We show that placing 9 hyperrooks is possible by construction.

Construction for 9 rooks:
\begin{table}[h]
\centering
\begin{tabular}{| c | c| c |}
\multicolumn{3}{c}{Top layer}\\
\hline
R& &\\
\hline
&R&\\
\hline
&&R\\
\hline
\end{tabular}
\quad
\begin{tabular}{| c | c| c |}
\multicolumn{3}{c}{Middle layer}\\
\hline
&R &\\
\hline
&&R\\
\hline
R&&\\
\hline
\end{tabular}
\quad
\begin{tabular}{| c | c| c |}
\multicolumn{3}{c}{Bottom layer}\\
\hline
& &R\\
\hline
R&&\\
\hline
&R&\\
\hline
\end{tabular}
\end{table}

\item % Sharp Maths Competition, Easter Egg for Liam
{\itshape Find all positive integers $a$, $b$ and $c$ satisfying 
$$a + b - c = 14$$
$$a^2 + b^2 - c^2 = 14.$$}


\end{enumerate}

\vfill
% ASCII art
\centering
\tiny
\begin{BVerbatim}
       ,----,
  ___.`      `,
  `===  D     :
    `'.      .'
       )    (                   ,
      /      \_________________/|
     /                          |
    |                           ;
    |               _____       /
    |      \       ______7    ,'
    |       \    ______7     /
     \       `-,____7      ,'   
^~^~^~^`\                  /~^~^~^~^
 ~^~^~^ `----------------' ~^~^~^
~^~^~^~^~^^~^~^~^~^~^~^~^~^~^~^~
\end{BVerbatim}
\end{document}
