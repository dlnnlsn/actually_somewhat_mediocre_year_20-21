\documentclass{article}

\usepackage{mathtools,amsfonts}
\usepackage{enumerate}
\usepackage{fullpage}
\usepackage{fancyvrb}
\usepackage{hyperref}


\begin{document}
\thispagestyle{empty}

\begin{center}
  \textbf{\Large Intermediate Test 3 Solutions}
  % LEVEL is Senior, Intermediate or Beginner
  % NUMBER is the test number: 1, 2, etc.
  \\ \vspace{1em}
  \textbf{\large January Camp 2021}
  \\ \vspace{1em}
  \textbf{\large Time: $2\frac{1}{2}$ hours}
\end{center}

\vspace{24pt}

\begin{enumerate}[1.]

\item % Ireland 2017 P4
{\itshape Find the smallest positive integer which only has $1$ and $2$ as digits and which is divisible by $99$.}


\item % Tim
{\itshape ${\Gamma}$ is a circle and ${AB}$ and ${CD}$ are ${\perp}$ chords that intersect inside ${\Gamma}$. ${P}$ is on ${\Gamma}$, with ${Q}$ diametrically opposite to ${P}$. Let the feet of the perpendiculars from ${P}$ to ${AB}$ and ${CD}$ be ${P'}$ and ${P''}$ respectively, and the feet of the perpendiculars from ${Q}$ to ${AB}$ and ${CD}$ be ${Q'}$ and ${Q''}$ respectively. Show that: $${ (P'P)^2+(P''P)^2+(Q'Q)^2+(Q''Q)^2 < d^2}$$ where ${d}$ is the diameter of ${\Gamma}$.}

\item % Taariq
{\itshape Let $q(x) = x^3 -x^2 -2x +1$ with roots $a$, $b$, and $c$.
Find $a^2 +b^2 + c^2$.}

Since $a,b,c$ are roots and $q(x)$ has no leading coefficient, we can factorise $q(x)$ into $q(x) = (x - a)(x - b)(x - c)$. Multiplying out gives $$q(x) = x^3 + x^2(-a - b - c) + x(ab + bc + ca) - abc$$ which means that $-a -b -c = -1$ and $ab + bc + ca = -2$. The first equation is equivalent to $a + b + c = 1$. These results could've been read directly from Viviani's. We finish with the following
\begin{align*}
a^2 + b^2 + c^2 &= (a + b + c)^2 - 2ab -2bc -2c\\
&= (a + b + c)^2 - 2(ab + bc + ca)\\
&= 1^2 -2(-2)\\
&= 5
\end{align*}


\item % Tim
{\itshape There is a book with $n$ chapters where chapter $i$ has $i$ pages.
The probability of opening the book in the same chapter twice in a row is $p$.
Is it possible for $p$ to be $1/k$ for some integer $k$?}

We note that the book contains
\[
  1 + 2 + 3 + \dotsb + n = \frac{n(n + 1)}{2}  
\]
pages.

To calculate the probability of opening the book in the same chapter twice, we count the number of ways to open it in the same chapter twice, and divide this by the total number of outcomes for opening the book twice.

Suppose that we initially open the book in chapter $i$. Then there are $i$ options for which page we initially opened it on, and there are $i$ ways to open it a second time in the same chapter. The total number of ways of opening it in chapter $i$ twice is thus $i^2$, and so the total number of ways of opening it in the same chapter twice is
\[
  1^2 + 2^2 + 3^2 + \dotsb + n^2 = \frac{n(n + 1)(2n + 1)}{6}.
\]

We see that the probability of opening it in the same chapter twice is
\[
  \frac{n(n + 1)(2n + 1)/6}{(n(n + 1)/2)^2} = \frac{4(2n + 1)}{6n(n + 1)}
\]

For this to be equal to $1/k$, we require
\[
  \frac{2(2n + 1)}{3n(n + 1)} = \frac{1}{k}.
\]
Since $2n + 1$ is relatively prime to both $n$ and $n + 1$, this would require that $2n + 1 \mid 3$. (We need the $2n + 1$ in the numerator to ``cancel''.) It then follows that $n = 1$, and so it is only possible for the probability to be of the form $\frac{1}{k}$ if the book has only one chapter.


\item % Ireland Book 2018 P1 (the Green one)
{\itshape Points $D$, $E$, and $F$ lie respectively on sides $BC$, $CA$, and $AB$ of triangle $ABC$ such that $BDEF$ is a parallelogram. Prove that the area of $BDEF$ is maximal when $D$, $E$, and $F$ are the midpoints of the sides.}


\end{enumerate}

\vfill
% ASCII art
\centering
\begin{BVerbatim}
  _          _
  \`.__..--'' `.
  ( _          ,\
 ( <_< < <   `','`.
  \ (_< < <    \   `.
   `. `----'   (  q _p
     `-._  _.-' `-(_''\
      (_'))--,      `._\
         `-._<
                 hjw
\end{BVerbatim}
\end{document}
