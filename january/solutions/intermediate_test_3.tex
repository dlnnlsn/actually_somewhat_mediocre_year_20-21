\documentclass{article}

\usepackage{mathtools,amsfonts}
\usepackage{enumerate}
\usepackage{fullpage}
\usepackage{fancyvrb}
\usepackage{hyperref}


\begin{document}
\thispagestyle{empty}

\begin{center}
  \textbf{\Large Intermediate Test 3 Solutions}
  % LEVEL is Senior, Intermediate or Beginner
  % NUMBER is the test number: 1, 2, etc.
  \\ \vspace{1em}
  \textbf{\large January Camp 2021}
  \\ \vspace{1em}
  \textbf{\large Time: $2\frac{1}{2}$ hours}
\end{center}

\vspace{24pt}

\begin{enumerate}[1.]

\item % Ireland 2017 P4
{\itshape Find the smallest positive integer which only has $1$ and $2$ as digits and which is divisible by $99$.}

Let this smallest positive integer be $N$. Note that $99 \mid N \iff 11 \mid N  \textrm{ and } 9 \mid N$. \par
Using the divisibility rules, we know that $9 \mid N$ means that the sum of the digits of $N$ will be a multiple of 9 
and that $11 \mid N$ means that the sum of every second digit minus the sum of every other digit will be a multiple of 11.
Let $A$ be the sum of every second digit and $B$ the sum of every other digit. \par
If the sum of $N$'s digits is equal to 9, we will have that $A + B = 9$ and either $A$ will be odd and $B$ even, or vice versa.
Since we have that $11 \mid A - B$ and $A + B = 9 < 11 \implies \mid A - B\mid = 0$ which is impossible since $A$ and $B$ have different parity. \par
Next we have the case of the sum of $N$'s digits being equal to 18 (the next multiple of 9). To minimise the number of digits we can make as many digits as possible equal to 2.
This would make the smallest option $N = 222222222$, however this would not be divisible by 11, as $\mid A - B\mid = 2$.
The next smallest possible option would $N = 1122222222$. Note that we place the 1's at the front of number to minimise the number.
In this case, the number would be divisible by 11 which would make $N = 1122222222$ the solution. 

\item % Tim
{\itshape ${\Gamma}$ is a circle and ${AB}$ and ${CD}$ are ${\perp}$ chords that intersect inside ${\Gamma}$. ${P}$ is on ${\Gamma}$, with ${Q}$ diametrically opposite to ${P}$. Let the feet of the perpendiculars from ${P}$ to ${AB}$ and ${CD}$ be ${P'}$ and ${P''}$ respectively, and the feet of the perpendiculars from ${Q}$ to ${AB}$ and ${CD}$ be ${Q'}$ and ${Q''}$ respectively. Show that: $${ (P'P)^2+(P''P)^2+(Q'Q)^2+(Q''Q)^2 < d^2}$$ where ${d}$ is the diameter of ${\Gamma}$.}

\item % Taariq
{\itshape Let $q(x) = x^3 -x^2 -2x +1$ with roots $a$, $b$, and $c$.
Find $a^2 +b^2 + c^2$.}

Since $a,b,c$ are roots and $q(x)$ has no leading coefficient, we can factorise $q(x)$ into $q(x) = (x - a)(x - b)(x - c)$. Multiplying out gives $$q(x) = x^3 + x^2(-a - b - c) + x(ab + bc + ca) - abc$$ which means that $-a -b -c = -1$ and $ab + bc + ca = -2$. The first equation is equivalent to $a + b + c = 1$. These results could've been read directly from Viviani's. We finish with the following
\begin{align*}
a^2 + b^2 + c^2 &= (a + b + c)^2 - 2ab -2bc -2c\\
&= (a + b + c)^2 - 2(ab + bc + ca)\\
&= 1^2 -2(-2)\\
&= 5
\end{align*}


\item % Tim
{\itshape There is a book with $n$ chapters where chapter $i$ has $i$ pages.
The probability of opening the book in the same chapter twice in a row is $p$.
Is it possible for $p$ to be $1/k$ for some integer $k$?}

Notice that the book has $\frac{n(n+1)}{2}$ pages ($1+2+3+...+n$). Now, the number of ways that we can land in the same chapter twice is $1^2+2^2+3^2+...+n^2$, since each chapter has $i$ pages that we could have landed in each time. This can be simplified as: $1^2+2^2+3^2+...+n^2 = \frac{n(n+1)(2n+1)}{6}$. The total number of ways to open the book twice is just $(\frac{n(n+1)}{2})^2$, since we can land on any page, then any page again. So $p = \frac{n(n+1)(2n+1)}{6}/(\frac{n(n+1)}{2})^2 = \frac{(2n+1)}{3}/(\frac{n(n+1)}{2})=\frac{2(2n+1)}{3n(n+1)}$. Since $n(n+1)$ is always divisible by 2, the 2 on the top will cancel. Now, we seek $n$ such that $2n+1$ will cancel i.e. since $2n+1$ is odd, we seek $n$ with $2n+1\:|\: 3n(n+1)$.
\begin{align*}
2n+1\:&|\: 3n^2+3n\\
2n+1\:&|\: 2(3n^2+3n)-3n(2n+1) = 3n\\
2n+1\:&|\: 2(3n)-3(2n+1)\\
2n+1\:&|\: -3\\
\end{align*}
Finally, we get $2n+1=1,3,-1,-3$ which gives $n=0,1,-1,-2$, none of which are valid numbers of chapters. So there is no $n>1$ giving $p=\frac{1}{k}$.


\item % Ireland Book 2018 P1 (the Green one)
{\itshape Points $D$, $E$, and $F$ lie respectively on sides $BC$, $CA$, and $AB$ of triangle $ABC$ such that $BDEF$ is a parallelogram. Prove that the area of $BDEF$ is maximal when $D$, $E$, and $F$ are the midpoints of the sides.}


\end{enumerate}

\vfill
% ASCII art
\centering
\begin{BVerbatim}
  _          _
  \`.__..--'' `.
  ( _          ,\
 ( <_< < <   `','`.
  \ (_< < <    \   `.
   `. `----'   (  q _p
     `-._  _.-' `-(_''\
      (_'))--,      `._\
         `-._<
                 hjw
\end{BVerbatim}
\end{document}
