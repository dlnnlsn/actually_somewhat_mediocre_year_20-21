\documentclass{article}

\usepackage{mathtools,amsfonts}
\usepackage{enumerate}
\usepackage{fullpage}
\usepackage{fancyvrb}
\usepackage{hyperref}


\begin{document}
\thispagestyle{empty}

\begin{center}
  \textbf{\Large Advanced Test 3}
  % LEVEL is Senior, Intermediate or Beginner
  % NUMBER is the test number: 1, 2, etc.
  \\ \vspace{1em}
  \textbf{\large January Camp 2021}
  \\ \vspace{1em}
  \textbf{\large Time: $2\frac{1}{2}$ hours}
\end{center}

\vspace{12pt}

\begin{enumerate}[1.]

\item % Tim
There is a book with $n$ chapters where chapter $i$ has $i$ pages.
The probability of opening the book in the same chapter twice in a row is $p$.
Is it possible for $p$ to be $1/k$ for some integer $k$?


\item % Ireland Book 2018 P1 (the Green one)
Points $D$, $E$, and $F$ lie respectively on sides $BC$, $CA$, and $AB$ of triangle $ABC$ such that $BDEF$ is a parallelogram. Prove that the area of $BDEF$ is maximal when $D$, $E$, and $F$ are the midpoints of the sides.


\item % Canada MO 2013, Q1
Find all polynomials $P$ with real coefficients such that $(x + 1)P(x - 1) - (x - 1)P(x)$ is constant.


\item % 


\item % Ukraine 2016-2017 Grade 11 Round 3 Q5
Let $ABC$ be an acute non-isosceles triangle with altitudes $BB_1$ and $CC_1$ intersecting at $H$.
The angle bisectors of $\angle B_1AC_1$ and $B_1HC_1$ intersect the line $B_1C_1$ at points $L_1$ and $L_2$, respectively.
Let $P$ and $Q$ be the second points of intersection of the circumcircles of triangles $AHL_1$ and $AHL_2$ with the line $B_1C_1$ respectively.
Prove that the points $B$, $C$, $P$, and $Q$ lie on a circle.

\end{enumerate}


\vfill
\begin{itemize}
	\item Submit your solutions at \url{https://forms.gle/M1L9KgbwzDxCKEjD9}.
	\item Submit each question in a single separate PDF file (with multiple pages if necessary).
	\item If you take photographs of your work, use a document scanner such as Office Lens to convert to PDF.
	\item If you have multiple PDF files for a question, combine them using software such as PDFsam.
\end{itemize}

\vfill
% ASCII art
\centering
\tiny
\begin{BVerbatim}
___§§§§
__8_()__§
_88____ §
888___ §____ $$$$s
____§___§___80_)$$$
_____§__§__88$$$$$$
_____§__§_8888$$$$$
_____§__§_____$$$$$s
_____§__§______$$$$$
_____§__§_______$$$$
_____§__§_______$$$$
_____§__§_______$$$$
_____§__§_______$$$$
____§___§_______$$$$
____§___§_______$$$$
____§____§______$$$$
___§______§§§§__$$$$$
___§__________§§§§$$$$$
___§______________§$$$$$$$
__ §_ §_____________§$$$$$$$$$
__§__§______________§§$$$$$$$$$
__§__§______§_________§§§§$$§$$$
__§___§_____§__§__________§§_§$$$$
__§____§_____§__§____________§$$$$$
___§____§_____§__§§§____§____§$$$$$$$___$
___§____§§_____§§§§§§§§§____§$$$$$$$$$$$
____§____§§§_____§§§§______§$$$$$$$$$$$
_____§____§_§§§§__§§§§____§$$$$$$$$$$$
______§____§§_§§§§§______§$$$$$$$$$$$
_______§§__§___§§______§§$$$$$$$$$$
_________§__§§§____§§§_$$$$$$$$$$
__________§__§__§§______$$$$$&
___________§§&§§_______&_____&
_____________&________&______&
____________&&______&&_______&&
_________&&&&&__&&&&&&___&&&&&&
_______&&&&_&_____&&&______&&&
\end{BVerbatim}

\end{document}
